\documentclass[preprint]{aastex}
\usepackage{fixltx2e}
\usepackage{amsmath}
\usepackage{verbatim}
\bibliographystyle{plain}
\begin{document}
\shorttitle{The RR Lyrae PL Relation in IRAC Ch. 1 \& 2}
\shortauthors{Durbin et al.}
\title{The RR Lyrae Period-Luminosity Relation in IRAC Channels 1 and 2}
\author{Meredith Durbin\altaffilmark{1}, Victoria Scowcroft\altaffilmark{2}, Wendy Freedman\altaffilmark{2}, Barry F. Madore\altaffilmark{2}, Andrew J. Monson\altaffilmark{2}, Mark Seibert\altaffilmark{2}, and Jeffrey Rich\altaffilmark{2}}
\affil{$^1$Pomona College, Claremont, CA, 91711}
\affil{$^2$Observatories of the Carnegie Institution of Washington, 813 Santa Barbara St., Pasadena, CA 91101, USA}
\begin{abstract}
We present new period-luminosity relations for RR Lyrae variables in 3.6 and 4.5 \micron\ derived from time-resolved IRAC data of $\omega$ Centauri. The sample consists of 36 RR Lyrae in 3.6 \micron\ and 37 in 4.5 \micron, 22 of which appear in both channels and have literature values for metallicities. We find no compelling evidence for a metallicity correlation in the residuals, based on a spread of 1.2 dex in [Fe/H].
\end{abstract}

\keywords{stars: variables: other --- stars: horizontal-branch ---
infrared: stars --- RR Lyrae variables --- globular clusters: individual(\objectname{$\omega$ Centauri} \objectname{NGC 5139})}

\section{Introduction}

The Carnegie-Hubble Program is an effort led by the Observatories of the Carnegie Institution of Washington to reduce the total systematic and statistical error in $H_0$ to $\pm 2\%$ or better, and the Carnegie RR Lyrae Program is a branch of the CHP focusing on distance calibration with RR Lyrae. The RR Lyrae distance scale does not extend as far as the Cepheid distance scale, but it is more precise than and rigorously independent of the Cepheid scale, which makes it an excellent candidate for constraining systematic errors in Cepheid distances as well as a valuable calibrator in its own right.

RR Lyrae are exceptional distance indicators in multiple wavelengths, whether by the $M_V-$[Fe/H] relation or by the period-luminosity relation in near- to mid-infrared wavelengths. They are known to yield distances at or below $\pm 2$\% error for single stars, and are fully decoupled from the Population I Cepheid distance scale [[I got this from the Spitzer proposal, is there something better to cite?]].
Here we present a mid-infrared of the RR Lyrae period-luminosity (PL) relation in the IRAC channels 1 and 2 centered on 3.6 and 4.5 \micron\ respectively, as well as a preliminary investigation into metallicity effects on the PL relation.

The advantages of the mid-infrared with regard to distance calibrations generally, and RR Lyrae specifically, have been previously outlined in \citet{2013ApJ...776..135M}. Line-of-sight extinction is universally reduced in the mid-IR by factors of up to 17 in 3.6 \micron, and up to 43 in 4.5 \micron\ \citep{2011AJ....142..192F}. RR Lyrae luminosities in the mid-IR are far less dependent on temperature than in the optical, which reduces the intrinsic width of the period-luminosity relation as well as the amplitude of variation in the light curves. There are also very few metallic or molecular transition lines in the mid-IR at typical RR Lyrae temperatures, so the effects of metallicity on luminosity should be minimized. Thus, as one moves from the optical to the mid-IR, the slope of the PL relation steepens and scattered is dramatically decreased; this phenomenon has been demonstrated in simulations by \citet{2004ApJS..154..633C}, and by several observational efforts, as illustrated in fig. 4 of \citet{2013ApJ...776..135M}. The slope should asymptotically approach the predicted slope of the period-radius relation, resulting in a slope from $-2.4$ to $-2.8$.

$\omega$ Cen in particular is ideal for calibrating the RR Lyrae period-luminosity-metallicity relation, as it contains 192 known RR Lyrae \citep{2004A&A...424.1101K} with a metallicity range spanning over 1.5 dex (Bono 2013, private communication); a metallicity spread this wide is not found in any other Milky Way globular cluster. As noted in \citet{2006MNRAS.372.1675S}, one of the advantages of using globular clusters to calibrate PL coefficients is that all stars in a cluster can be considered to be at the same distance from Earth and subject to the same line-of-sight extinction effect across the cluster, and thus we can be confident that the period-luminosity relations we see are intrinsic to the stars.

\section{Observations \& Data Reduction}
[Vicky says things]

PSF photometry was performed in PyRAF 2.0 using the \textsc{daophot} package. Initial aperture photometry was performed with an aperture radius of 3.6\arcsec and annulus radius of 4.8\arcsec, whereas the standard IRAC aperture is 6\arcsec; a correction factor of 1.12841 and 1.12738 for 3.6 and 4.5 \micron\ respectively was applied to the instrumental magnitudes. We also calibrate the magnitudes from images in counts (required for \textsc{daophot}) back to flux magnitudes by performing aperture photometry on a set of bright, isolated stars in the flux images, finding the mean difference between the aperture flux magnitudes and PSF counts magnitudes of the same stars, and subtracting said difference from all PSF magnitudes. We also correct for location in the frame.

The primary limiting factor in this data is crowding: 77 RR Lyrae out of 192 were rejected due to crowding. To decide which stars to reject, a $K$-band image from the Magellan telescope at Las Campanas Observatory was used, as it provided a full view of the entire cluster, and the most crowded regions were more obvious than in the Spitzer data, although the bandpasses are close enough that they are still comparable. The selection of which stars to reject was made on a primarily visual basis.


\section{Period-Luminosity Relations}

It is common practice to convert the RRc periods to fundamental mode periods (to ``fundamentalize" them) using the ratio observed in double mode RR Lyrae, where $P_1/P_0 = 0.74432 \pm 0.00003$ or $\log P_0 = \log P_1 + 0.128$ \citep{1996AJ....112.2026W}, such that the types can be combined for a larger sample size, as done in \citet{2004ApJ...610..269D}. \citet{2014MNRAS.440L..96K} present separate relations for each type, arguing that the combination of the types is physically inappropriate, whereas \citet{2014MNRAS.439.3765D} argue that RRc stars exhibit a $K$-band period-luminosity relation that differs from RRab's only by the period shift, and that this should not change in longer wavelengths. Here we have chosen to fundamentalize the RRc periods to maximize our sample sizes; we see no obvious evidence that the types should be separated.

We present PL relations of the form \begin{equation}m = \alpha_\lambda(\pm \sigma_{\alpha_\lambda}) \times \log P + \beta_\lambda(\pm \sigma_{\beta_\lambda})\end{equation}
where $m_\lambda$ is the apparent magnitude in wavelength $\lambda$, $\alpha_\lambda$ is the slope in the same wavelength, $\sigma_{\alpha_\lambda}$ is the error in the slope, $\beta_\lambda$ is the apparent zero point, and $\sigma_{\beta_\lambda}$ is the error in the zero point; we also include formal scatter $\sigma_\lambda$, the standard deviation of the residuals $m_{\text{observed}} - m_{\text{predicted}}$. Here we have weighted the linear fits by individual magnitude errors. See Table~\ref{tab:plparams} for the PL parameters we have derived.

These slopes agree within uncertainty with the slopes of the $W1$ (3.4 \micron) and $W2$ (4.6 \micron) absolute PL relations derived by \citet{2013ApJ...776..135M}, \citet{2014MNRAS.440L..96K}, and \citet{2014MNRAS.439.3765D}; our [4.5] slope in particular is in excellent agreement with Klein et al. (It should be noted that Klein et al. present separate PL relations for RRab's and RRc's; here we are comparing to their RRab slope.) Our formal scatter measurements agree with those of Madore et al. 

The scatter in these PL relations is higher than we expect the intrinsic scatter in these bands to be. Some factors that may contribute to this are crowding (although the most crowded stars having been removed), misidentification of stars in \textsc{daomatch}, and image artifacts present in certain frames. Several of the light curves have one or two data points which are several sigma (up to approximately half a magnitude) brighter or dimmer than the typical range of the rest of the stars; this could affect the mean magnitudes and lead to increased scatter overall.

\section{Distance Moduli}

Madore et al. (2014, in prep) have found that there is effectively no offset between magnitudes measured in $W1$ vs. [3.6], or between $W2$ vs. [4.5]. Therefore, it is acceptable to obtain distance moduli $\mu$ by combining absolute and apparent PL zero points from $W1$ and [3.6] respectively, and $W2$ and [4.5] and thus obtaining $\mu = m - M$. However, this method is only viable when the slopes are in agreement. We thus recalibrate our zero points by constraining the slopes of our PL fits to the slopes presented in Madore et al (2013), Klein et al (2014), Dambis et al (2014), and Neeley (discussed in other section of this paper?) respectively, and leaving only the zero point as a free parameter. See Table~\ref{tab:distance} for our recalibrated zero points and distance moduli.

Our values are all slightly lower than previous distance measurements using RR Lyrae in the near-IR \citep{2006ApJ...652..362D} and the eclipsing binary OGLEGC17 \citep{2001AJ....121.3089T}, but higher than the distances measured by dynamical modeling \citep{2013MNRAS.436.2598W, 2006A&A...445..513V}. It should be noted that all of these results have their caveats: Madore et al. have large error bars due to small sample size; Klein et al. include only RRab stars in their fit, whereas we include fundamentalized RRc stars; and Dambis et al. include metallicity terms, which we do not. Nevertheless, these results are promising foundations for further analysis.


\section{Metallicity}

In order to put constraints on the value of a metallicity term in either channel, here we investigate the correlation between individual RR Lyrae metallicities and their PL residuals. Given that the $\pm2\sigma$ width of each PL relation is about 0.4 mag, any metallicity effect must be contained within that range, so we expect it to be intrinsically small. If there is any correlation between [Fe/H] and the PL residuals, we expect it to be a linear one, similar to the metallicity terms of order $\gamma \log Z$ in optical and near-IR PL relations. We present metallicity-residual relations of the form \begin{equation}\Delta m_\lambda = \alpha_\lambda(\pm \sigma_{\alpha_\lambda}) \times [\text{Fe/H}] + \beta_\lambda(\pm \sigma_{\beta_\lambda})\end{equation}
where $m_\lambda$ is the apparent magnitude in wavelength $\lambda$, $\alpha_\lambda$ is the slope in the same wavelength, $\sigma_{\alpha_\lambda}$ is the error in the slope, $\beta_\lambda$ is the apparent zero point, and $\sigma_{\beta_\lambda}$ is the error in the zero point; again $\sigma_\lambda$ is the formal scatter. Relations were fit without weighting individual errors in either [Fe/H] or $\Delta m$; although there are large errors in both variables, they are consistent enough that weighting by errors would result in a fit heavily skewed by the few data points with low errors.

We use both photometric \citep{2000AJ....119.1824R}, spectroscopic \citep{2006ApJ...640L..43S}, and combined metallicities for comparison; while the Rey catalog contains 131 RR Lyrae metallicites as opposed to the 74 in Sollima, spectroscopic metallicities are much more reliable than photometric metallicities, as photometric metallicities rely on certain assumptions about the chemical abundances of stars whereas spectroscopic metallicities provide actual abundances. For stars appearing in our sample, there is a metallicity spread of up to 1.19 dex using the Rey catalog, and 0.76 dex using Sollima. When we combine the catalogs, we favor the spectroscopic metallicities when both are available.

See Table~\ref{tab:photmetallicity} for photometric metallicity-residual relation parameters, Table~\ref{tab:specmetallicity} for spectroscopic, and Table~\ref{tab:bothmetallicity} for combined.

For photometric metallicities only, the slopes are $1\sigma$ away from zero in 3.6 \micron, and $2\sigma$ in 4.5 \micron. These could be construed as evidence for a weak metallicity relationship, but it should be noted that the formal scatter in each case is of the same width as that of the PL relations themselves; the correlation is weak at best.

For spectroscopic metallicities only, both slopes are within $1\sigma$ of zero; however, the sample sizes and metallicity range are smaller (22 vs. 32 RR Lyrae in 3.6 \micron, and 18 vs. 33 in 4.5 \micron), suggesting that this is unlikely to be due to the improved accuracy of spectroscopic metallicities.

The combined metallicity relations are almost identical in slope and scatter to the photometric metallicity relations, as there are several high-metallicity points from the photometric metallicities that affect the fits strongly, particularly in 4.5 \micron.
 
While these relations do not completely rule out the possibility of a metallicity term in the PL relations, the evidence in favor of one is not extremely compelling, particularly in [3.6], where the slope is $1\sigma$ from zero at most. There is up to $3\sigma$ evidence for a metallicity correlation in [4.5], but it should be noted that in all cases the formal scatter of each relation is of the same width as the PL relation itself, suggesting that the metallicity relation is dominated by scatter. At present, more data is required to establish the metallicity terms.

\section{Discussion}

The PL slopes we find here generally agree within error with WISE slopes, although our [3.6] slope tends to run steeper than others. Similar to Dambis et al, we find that the slope becomes shallower and the metallicity correlation stronger when moving from [3.6] to [4.5]; this is contrary to the predictions made by Madore et al. that the period-luminosity slope will asymptotically approach the period-radius slope as one moves farther into the infrared. Further study is required to determine whether this is truly an intrinsic effect or simply a coincidence of the data sets. It should be noted that in the case of Cepheid variables, Scowcroft et al. (2011) have shown that in [4.5] there is absorption due to a CO bandhead that appears at at 4.65 \micron, which flattens the slope of the PL relation. However, this effect is due to the low temperature of Cepheid atmospheres, and disappears in the hottest, shortest-period Cepheids, as the CO dissociates at temperatures above 6000 K \citep{2012ApJ...759..146M}. As all RR Lyrae have effective temperatures higher than 6000 K, we expect no such CO absorption. However, if it proves to be true that the RR Lyrae PL slope is intrinsically flatter in [4.5] than in [3.6], there may be reason to look for a previously unknown effect similar to that of the CO bandhead in RR Lyrae.

\section{Conclusions \& Future Work}
We have presented calibrations of period-luminosity relations in 3.6 and 4.5 \micron\ using 36 and 37 $\omega$ Centauri RR Lyrae respectively, the parameters of which are in fair or good agreement with previous WISE results. %waffling about distance modulus here

For the immediate future, the next steps are to continue doing work of this type on more clusters, and to further investigate the question of the metallicity term. This could conceivably be done in a manner similar to that of Dambis et al (2014) in which they use multiple clusters to derive a metallicity term based on the slopes of each cluster�s PL relation vs. the mean cluster metallicity; it could also be done with individual RR Lyrae metallicities in multiple clusters. The question of whether the PL slope in 4.5 \micron\ is in fact intrinsically shallower may also be worth investigating.

It is expected that the observations of GAIA and JWST will provide the next phase of this project; parallaxes from GAIA will increase the number of calibrators for the absolute PL relations, and JWST will be able to observe RR Lyrae in galaxies more distant than previously possible.

\acknowledgements
This work is based on observations made with the Spitzer Space Telescope, which is operated by the Jet Propulsion Laboratory, California Institute of Technology under a contract with NASA. Support for this work was provided by NASA through an award issued by JPL/Caltech.

\bibliography{Durbin_paper}

\clearpage

\appendix

\begin{deluxetable}{cccccc}
\centering
\tablecaption{Period-Luminosity Relation Parameters\label{tab:plparams}}
\tablewidth{0pt}
\tablehead{
\colhead{$\lambda$} & \colhead{$\alpha_\lambda$} & \colhead{$\sigma_{\alpha_\lambda}$} & \colhead{$\beta_\lambda$} & \colhead{$\sigma_{\beta_\lambda}$} & \colhead{$\sigma_\lambda$} 
}
\startdata
$[3.6]$ & $-2.50$ & 0.09 & 12.38 & 0.02 & 0.10 \\
$[4.5]$ & $-2.40$ & 0.12 & 12.42 & 0.03 & 0.09 \\
\enddata
\end{deluxetable}

\begin{deluxetable}{ccccc}
\centering
\tablecaption{Recalibrated Zero Points \& Distance Moduli\label{tab:distance}}
\tablewidth{0pt}
\tablehead{
\colhead{Relation} & \colhead{$\beta_{[3.6]}$} & \colhead{$\mu$ [3.6]} & \colhead{$\beta_{[4.5]}$} & \colhead{$\mu$ [4.5]}
}
\startdata
Madore & $12.40 \pm 0.02$ & $13.66 \pm 0.25$ & $12.38 \pm 0.03$ & $13.64 \pm 0.23$ \\
Klein & $12.41 \pm 0.02$ & $13.52 \pm 0.02$ & $12.42 \pm 0.03$ & $13.54 \pm 0.03$ \\
Dambis & $12.41 \pm 0.02$ & $13.56 \pm 0.08$ & $12.45 \pm 0.03$ & $13.60 \pm 0.08$ \\
Neeley & $12.44 \pm 0.02$ & $13.64 \pm 0.08$ & $12.41 \pm 0.03$ & $13.67 \pm 0.09$
\enddata
\end{deluxetable}

\begin{deluxetable}{cccccc}
\centering
\tablecaption{Photometric Metallicity-Residual Relation Parameters\label{tab:photmetallicity}}
\tablewidth{0pt}
\tablehead{
\colhead{$\lambda$} & \colhead{$\alpha_\lambda$} & \colhead{$\sigma_{\alpha_\lambda}$} & \colhead{$\beta_\lambda$} & \colhead{$\sigma_{\beta_\lambda}$} & \colhead{$\sigma_\lambda$} 
}
\startdata
$[3.6]$ & $-0.06$ & 0.05 & $- 0.12$ & 0.07 & 0.11 \\
$[4.5]$ & $-0.10$ & 0.04 & $- 0.18$ & 0.06 & 0.09
\enddata
\end{deluxetable}

\begin{deluxetable}{cccccc}
\centering
\tablecaption{Spectroscopic Metallicity-Residual Relation Parameters\label{tab:specmetallicity}}
\tablewidth{0pt}
\tablehead{
\colhead{$\lambda$} & \colhead{$\alpha_\lambda$} & \colhead{$\sigma_{\alpha_\lambda}$} & \colhead{$\beta_\lambda$} & \colhead{$\sigma_{\beta_\lambda}$} & \colhead{$\sigma_\lambda$} 
}
\startdata
$[3.6]$ & $-0.05$ & 0.09 & $- 0.08$ & 0.14 & 0.09 \\
$[4.5]$ & $-0.04$ & 0.14 & $- 0.07$ & 0.06 & 0.09
\enddata
\end{deluxetable}

\begin{deluxetable}{cccccc}
\centering
\tablecaption{Combined Metallicity-Residual Relation Parameters\label{tab:bothmetallicity}}
\tablewidth{0pt}
\tablehead{
\colhead{$\lambda$} & \colhead{$\alpha_\lambda$} & \colhead{$\sigma_{\alpha_\lambda}$} & \colhead{$\beta_\lambda$} & \colhead{$\sigma_{\beta_\lambda}$} & \colhead{$\sigma_\lambda$} 
}
\startdata
$[3.6]$ & $-0.05$ & 0.05 & $- 0.10$ & 0.07 & 0.10 \\
$[4.5]$ & $-0.10$ & 0.04 & $- 0.17$ & 0.08 & 0.09 
\enddata
\end{deluxetable}

\begin{deluxetable}{c c l l l l l l l l l l l}
\centering
\tabletypesize{\footnotesize}
\tablecaption{$\omega$ Cen data. 
Column headings are defined as follows.
ID: Star ID from OGLE. 
Type: pulsational mode. 
m$_{[\lambda]}$: Apparent magnitude in each channel. 
$\sigma_m$: magnitude errors.
P: period in days, from Kaluzny et al. (2004). 
$\Delta[\lambda]$: magnitude residuals.
[Fe/H]$_p$: Photometric metallicity from Rey et al. (2000).
$\sigma_p$: photometric metallicity error.
[Fe/H]$_s$: Spectroscopic metallicity from Kaluzny et al. (2006).
$\sigma_s$: spectroscopic metallicity error.
}
\tablewidth{0pt}
\tablehead{
\colhead{ID} & \colhead{Type} & \colhead{$m_{[3.6]}$} & \colhead{$\sigma_{m}$} & \colhead{$m_{[4.5]}$} & \colhead{$\sigma_{m}$} & \colhead{P} & \colhead{$\Delta[3.6]$} & \colhead{$\Delta[4.5]$} & \colhead{[Fe/H]$_p$} & \colhead{$\sigma_{p}$} & \colhead{[Fe/H]$_s$} & \colhead{$\sigma_{s}$}
}
\startdata
3 & ab & $12.66$ & $0.046$ & $12.619$ & $0.071$ & $0.841$ & $-0.091$ & $-0.022$ & $-1.54$ & $0.05$ & --- & --- \\
4 & ab & $13.063$ & $0.086$ & $13.113$ & $0.08$ & $0.627$ & $-0.175$ & $-0.21$ & $-1.74$ & $0.05$ & --- & --- \\
5 & ab & $13.285$ & $0.067$ & $13.279$ & $0.042$ & $0.515$ & $-0.184$ & $-0.171$ & $-1.35$ & $0.08$ & $-1.24$ & $0.11$ \\
9 & ab & $13.265$ & $0.045$ & --- & --- & $0.523$ & $-0.18$ & --- & $-1.49$ & $0.06$ & --- & --- \\
10 & c & $13.144$ & $0.047$ & $13.083$ & $0.075$ & $0.375$ & $-0.018$ & $0.049$ & $-1.66$ & $0.1$ & --- & --- \\
11 & ab & $12.94$ & $0.106$ & --- & --- & $0.565$ & $0.061$ & --- & $-1.67$ & $0.13$ & $-1.61$ & $0.22$ \\
12 & c & $13.18$ & $0.075$ & $13.2$ & $0.105$ & $0.387$ & $-0.087$ & $-0.101$ & $-1.53$ & $0.14$ & --- & --- \\
13 & ab & --- & --- & $12.761$ & $0.034$ & $0.669$ & --- & $0.075$ & $-1.91$ & $0.0$ & --- & --- \\
14 & c & --- & --- & $13.111$ & $0.05$ & $0.377$ & --- & $0.015$ & $-1.71$ & $0.13$ & --- & --- \\
15 & ab & $12.605$ & $0.102$ & $12.69$ & $0.099$ & $0.811$ & $0.004$ & $-0.054$ & $-1.64$ & $0.39$ & $-1.68$ & $0.18$ \\
18 & ab & $12.832$ & $0.05$ & --- & --- & $0.622$ & $0.066$ & --- & $-1.78$ & $0.28$ & --- & --- \\
20 & ab & $12.955$ & $0.067$ & $12.955$ & $0.053$ & $0.616$ & $-0.047$ & $-0.033$ & --- & --- & $-1.52$ & $0.34$ \\
21 & c & $13.308$ & $0.094$ & $13.169$ & $0.068$ & $0.381$ & $-0.198$ & $-0.054$ & $-0.9$ & $0.11$ & --- & --- \\
23 & ab & $12.989$ & $0.059$ & --- & --- & $0.511$ & $0.122$ & --- & $-1.08$ & $0.14$ & $-1.35$ & $0.58$ \\
30 & c & $12.938$ & $0.032$ & $13.05$ & $0.092$ & $0.404$ & $0.106$ & $0.003$ & $-1.75$ & $0.17$ & $-1.62$ & $0.28$ \\
33 & ab & --- & --- & $12.861$ & $0.092$ & $0.602$ & --- & $0.085$ & $-2.09$ & $0.23$ & $-1.58$ & $0.42$ \\
34 & ab & --- & --- & $12.677$ & $0.072$ & $0.734$ & --- & $0.062$ & $-1.71$ & $0.0$ & --- & --- \\
40 & ab & $12.88$ & $0.036$ & $12.967$ & $0.063$ & $0.634$ & $-0.004$ & $-0.075$ & $-1.6$ & $0.08$ & $-1.62$ & $0.19$ \\
44 & ab & $13.065$ & $0.023$ & $13.015$ & $0.043$ & $0.568$ & $-0.069$ & $-0.008$ & $-1.4$ & $0.12$ & $-1.29$ & $0.35$ \\
45 & ab & --- & --- & $12.876$ & $0.045$ & $0.589$ & --- & $0.092$ & $-1.78$ & $0.25$ & --- & --- \\
47 & c & $13.107$ & $0.111$ & $13.064$ & $0.059$ & $0.485$ & $-0.26$ & $-0.201$ & $-1.58$ & $0.31$ & --- & --- \\
49 & ab & --- & --- & $12.899$ & $0.053$ & $0.605$ & --- & $0.043$ & $-1.98$ & $0.11$ & --- & --- \\
50 & c & --- & --- & $13.151$ & $0.07$ & $0.386$ & --- & $-0.05$ & $-1.59$ & $0.19$ & --- & --- \\
51 & ab & $13.012$ & $0.034$ & --- & --- & $0.574$ & $-0.028$ & --- & $-1.64$ & $0.21$ & $-1.84$ & $0.23$ \\
54 & ab & $12.627$ & $0.037$ & --- & --- & $0.773$ & $0.034$ & --- & $-1.66$ & $0.12$ & $-1.8$ & $0.23$ \\
56 & ab & --- & --- & $13.07$ & $0.049$ & $0.568$ & --- & $-0.064$ & $-1.26$ & $0.15$ & --- & --- \\
58 & c & $13.255$ & $0.067$ & $13.196$ & $0.137$ & $0.37$ & $-0.114$ & $-0.05$ & $-1.37$ & $0.18$ & $-1.91$ & $0.31$ \\
59 & ab & $12.945$ & $0.054$ & --- & --- & $0.519$ & $0.149$ & --- & $-1.0$ & $0.28$ & --- & --- \\
64 & c & --- & --- & $13.16$ & $0.05$ & $0.344$ & --- & $0.06$ & $-1.46$ & $0.23$ & --- & --- \\
66 & c & $13.103$ & $0.059$ & --- & --- & $0.407$ & $-0.067$ & --- & $-1.68$ & $0.34$ & --- & --- \\
67 & ab & $13.211$ & $0.062$ & --- & --- & $0.564$ & $-0.209$ & --- & $-1.1$ & $0.0$ & $-1.19$ & $0.23$ \\
68 & c & $12.74$ & $0.052$ & --- & --- & $0.535$ & $0.001$ & --- & $-1.6$ & $0.01$ & --- & --- \\
82 & c & $13.206$ & $0.063$ & $13.166$ & $0.037$ & $0.336$ & $0.041$ & $0.08$ & $-1.56$ & $0.2$ & $-1.71$ & $0.56$ \\
94 & c & $13.598$ & $0.057$ & $13.582$ & $0.083$ & $0.254$ & $-0.047$ & $-0.044$ & $-1.0$ & $0.11$ & --- & --- \\
95 & c & --- & --- & $12.983$ & $0.058$ & $0.405$ & --- & $0.069$ & $-1.84$ & $0.55$ & --- & --- \\
97 & ab & $12.786$ & $0.073$ & $12.688$ & $0.047$ & $0.692$ & $-0.005$ & $0.113$ & $-1.56$ & $0.37$ & $-1.74$ & $0.17$ \\
102 & ab & $12.804$ & $0.057$ & $12.899$ & $0.051$ & $0.691$ & $-0.022$ & $-0.097$ & $-1.84$ & $0.13$ & $-1.65$ & $0.16$ \\
103 & c & $13.347$ & $0.076$ & $13.462$ & $0.098$ & $0.329$ & $-0.077$ & $-0.194$ & $-1.92$ & $0.11$ & $-1.78$ & $0.27$ \\
107 & ab & --- & --- & $13.272$ & $0.086$ & $0.514$ & --- & $-0.162$ & $-1.36$ & $0.11$ & --- & --- \\
115 & ab & $12.855$ & $0.043$ & $12.825$ & $0.067$ & $0.63$ & $0.027$ & $0.073$ & $-1.87$ & $0.01$ & $-1.64$ & $0.32$ \\
117 & c & $12.856$ & $0.043$ & $13.034$ & $0.104$ & $0.422$ & $0.143$ & $-0.025$ & $-1.68$ & $0.25$ & --- & --- \\
120 & ab & $12.875$ & $0.027$ & $12.981$ & $0.062$ & $0.549$ & $0.158$ & $0.062$ & $-1.39$ & $0.06$ & $-1.15$ & $0.16$ \\
121 & c & $13.309$ & $0.026$ & $13.334$ & $0.068$ & $0.304$ & $0.045$ & $0.016$ & $-1.46$ & $0.13$ & $-1.83$ & $0.4$ \\
122 & ab & $12.849$ & $0.038$ & $12.786$ & $0.084$ & $0.635$ & $0.025$ & $0.105$ & $-2.02$ & $0.18$ & $-1.79$ & $0.21$ \\
169 & c & $13.32$ & $0.056$ & $13.392$ & $0.058$ & $0.319$ & $-0.018$ & $-0.092$ & --- & --- & $-1.65$ & $0.19$ \\
274 & c & $13.412$ & $0.031$ & --- & --- & $0.311$ & $-0.083$ & --- & --- & --- & --- & --- \\
291 & c & --- & --- & $13.196$ & $0.067$ & $0.334$ & --- & $0.056$ & --- & --- & --- & --- \\
357 & c & $13.361$ & $0.042$ & $13.279$ & $0.055$ & $0.298$ & $0.016$ & $0.093$ & --- & --- & $-1.64$ & $0.99$ \\
\enddata
\end{deluxetable}

\begin{figure}
\figurenum{1}
\epsscale{1.1}
\plottwo{PL_3p6.eps}{PL_4p5.eps}
\caption{The RR Lyrae period-luminosity relation for $\omega$ Cen in [3.6] and [4.5]. The blue data points are RRab's, the green are RRc's with fundamentalized periods, and the flanking lines are 1 and 2$\sigma$.}
\end{figure}

\begin{figure}
\centering
\figurenum{2}
\epsscale{1.1}
\plottwo{Metallicity_3p6_Rey.eps}{Metallicity_4p5_Rey.eps}
\caption{Metallicity vs. PL residuals for $\omega$ Cen in [3.6] and [4.5] using photometric metallicities from Rey et al. (2000). The flanking lines are 1 and 2 standard deviations.}
\end{figure}

\begin{figure}
\centering
\figurenum{3}
\epsscale{1.1}
\plottwo{Metallicity_3p6_Sollima.eps}{Metallicity_4p5_Sollima.eps}
\caption{Metallicity vs. PL residuals for $\omega$ Cen in [3.6] and [4.5] using spectroscopic metallicities from Sollima et al. (2006). The flanking lines are 1 and 2 standard deviations.}
\end{figure}

\begin{figure}
\centering
\figurenum{4}
\epsscale{1.1}
\plottwo{Metallicity_3p6.eps}{Metallicity_4p5.eps}
\caption{Metallicity vs. PL residuals for $\omega$ Cen in [3.6] and [4.5] using combined metallicity catalogs, favoring spectroscopic when both are available. The flanking lines are 1 and 2 standard deviations.}
\end{figure}

\end{document}

% mnras_template.tex
%
% LaTeX template for creating an MNRAS paper
%
% v3.0 released 14 May 2015
% (version numbers match those of mnras.cls)
%
% Copyright (C) Royal Astronomical Society 2015
% Authors:
% Keith T. Smith (Royal Astronomical Society)

% Change log
%
% v3.0 May 2015
%    Renamed to match the new package name
%    Version number matches mnras.cls
%    A few minor tweaks to wording
% v1.0 September 2013
%    Beta testing only - never publicly released
%    First version: a simple (ish) template for creating an MNRAS paper

%%%%%%%%%%%%%%%%%%%%%%%%%%%%%%%%%%%%%%%%%%%%%%%%%%
% Basic setup. Most papers should leave these options alone.
\documentclass[a4paper,fleqn,usenatbib]{mnras}

% MNRAS is set in Times font. If you don't have this installed (most LaTeX
% installations will be fine) or prefer the old Computer Modern fonts, comment
% out the following line
\usepackage{newtxtext,newtxmath}
% Depending on your LaTeX fonts installation, you might get better results with one of these:
%\usepackage{mathptmx}
%\usepackage{txfonts}

% Use vector fonts, so it zooms properly in on-screen viewing software
% Don't change these lines unless you know what you are doing
\usepackage[T1]{fontenc}
\usepackage{ae,aecompl}

%%%%% AUTHORS - PLACE YOUR OWN PACKAGES HERE %%%%%

% Only include extra packages if you really need them. Common packages are:
\usepackage{graphicx}	% Including figure files
\usepackage{amsmath}	% Advanced maths commands
\usepackage{amssymb}	% Extra maths symbols
\usepackage{longtable}
\usepackage{pdflscape}
\usepackage{xspace}
\usepackage{verbatim}

%%%%%%%%%%%%%%%%%%%%%%%%%%%%%%%%%%%%%%%%%%%%%%%%%%

%%%%% AUTHORS - PLACE YOUR OWN COMMANDS HERE %%%%%

% Please keep new commands to a minimum, and use \newcommand not \def to avoid
% overwriting existing commands. Example:
\newcommand{\ho}{$H_{0}$\xspace}

%%%%%%%%%%%%%%%%%%%%%%%%%%%%%%%%%%%%%%%%%%%%%%%%%%

%%%%%%%%%%%%%%%%%%% TITLE PAGE %%%%%%%%%%%%%%%%%%%

% Title of the paper, and the short title which is used in the headers.
% Keep the title short and informative.
\title[Mid--IR RRL PL Relation in $\omega$ Cen]{The Carnegie RR Lyrae Program: The Mid--Infrared RR Lyrae Period--Luminosity Relation in $\omega$~Cen}

% The list of authors, and the short list which is used in the headers.
% If you need two or more lines of authors, add an extra line using \newauthor
\author[M.~J.~Durbin et al.]{Meredith~J.~Durbin$^{1,2}$\thanks{E-mail: mdurbin@stsci.edu}
Victoria Scowcroft$^{3}$
Wendy Freedman$^{4}$
Barry Madore$^{3}$
\newauthor Gurtina Besla$^{5}$ 
Giuseppe Bono$^{6, 7}$
Maria--Rosa Cioni$^{8, 9, 10}$
Gisella Clementini$^{11}$
\newauthor Kathryn Johnston$^{12}$
Nitya Kallivayalil$^{13}$
Juna Kollmeier$^{3}$
David Law$^{2}$
Steve Majewski$^{13}$
\newauthor Roeland van der Marel$^{2}$
Massimo Marengo$^{14}$
Andrew~J.~Monson$^{15}$
David Nidever$^{16}$ 
\newauthor
Grzegorz Pietrzynski$^{17, 18}$
George Preston$^{3}$
Mark Seibert$^{3}$
Horace Smith$^{19}$
\newauthor Igor Soszynski$^{17}$
Ian Thompson$^{3}$
Andrzej Udalski$^{17}$
\\
% List of institutions
$^1$ Pomona College, Claremont, CA 91711, USA \\
$^2$ Space Telescope Science Institute, 3700 San Martin Drive, Baltimore, MD 21218, USA \\
$^3$ Observatories of the Carnegie Institution of Washington, 813 Santa Barbara St., Pasadena, CA 91101, USA \\
$^4$ Department of Astronomy and Astrophysics, University of Chicago, 5640 S Ellis Ave, Chicago, IL 60637, USA \\
$^5$ Department of Astronomy and Steward Observatory, University of Arizona, 933 North Cherry Avenue,   Tucson, AZ 85721, USA \\
$^6$ Univ. Roma ``Tor Vergata", Via della Ricerca Scientifica, 1 - 00133, Roma, Italy \\
$^7$ INAF-OAR, via Frascati 33 - 00040, Monte Porzio Catone (RM), Italy \\
$^8$ Universtat Potsdam, Institut fur Physik und Astronomie, Karl-Liebknecht-Str. 24/25, 14476 Potsdam, Germany \\
$^9$ Leibniz-Institut fur Astrophysik Potsdam, An der Sternwarte 16, 14482 Potsdam, Germany \\
$^{10}$ University of Hertfordshire, Physics, Astronomy and Mathematics, College Lane, Hatfield AL10 9AB, United Kingdom \\
$^{11}$ INAF - Osservatorio Astronomico, Via Ranzani n. 1, 40127 Bologna, Italy \\
$^{12}$ Department of Astronomy, Columbia University, New York, NY 10027, USA  \\
$^{13}$ Department of Astronomy, University of Virginia, Charlottesville, VA 22904-0818, USA \\
$^{14}$ Department of Physics and Astronomy, Iowa State University, Ames, IA, USA \\
$^{15}$ Department of Astronomy and Astrophysics, The Pennsylvania State University, 403 Davey Lab, University Park, PA, 16802, USA \\
$^{16}$ Department of Astronomy, University of Michigan, Ann Arbor, MI 48109, USA \\
$^{17}$ Warsaw University Observatory Al. Ujazdowskie 4, 00-478 Warszawa, Poland \\
$^{18}$ Departamento de Astronomia, Universidad de Concepcion, Casilla 160-C, Chile \\
$^{19}$ Department of Physics and Astronomy, Michigan State University, East Lansing, MI, USA 48824 \\
}

% These dates will be filled out by the publisher
\date{Accepted XXX. Received YYY; in original form ZZZ}

% Enter the current year, for the copyright statements etc.
\pubyear{2015}

% Don't change these lines
\begin{document}
\label{firstpage}
\pagerange{\pageref{firstpage}-\pageref{lastpage}}
\maketitle

% Abstract of the paper
\begin{abstract}
Something something metallicity
%We present new period-luminosity relations for RR Lyrae variables in 3.6 and 4.5 \micron\ derived from time-resolved IRAC data of $\omega$~Cen. The sample consists of 36 RR Lyrae in 3.6 \micron\ and 37 in 4.5 \micron, 22 of which appear in both channels and have literature values for metallicities. We find no compelling evidence for a metallicity correlation in the residuals, based on a spread of 1.2 dex in [Fe/H].
\end{abstract}

% Select between one and six entries from the list of approved keywords.
% Don't make up new ones.
\begin{keywords}
keyword1 - keyword2 - keyword3
\end{keywords}

%%%%%%%%%%%%%%%%%%%%%%%%%%%%%%%%%%%%%%%%%%%%%%%%%%

%%%%%%%%%%%%%%%%% BODY OF PAPER %%%%%%%%%%%%%%%%%%

%% IMPORTANT NOTES FROM VS:

% MNRAS is a UK journal - change your spellcheck language in your editor to British English. 
% Correct plural of RR Lyrae is RR Lyrae variables (technically, singular should be RR Lyrae variable, as RR Lyrae itself is a named object)
% Use 1 dash - for a minus sign in math mode
% Use 2 dashes -- to hyphenate words
% Use 3 dashes --- to put a dash between parts of a sentence or to denote a minus sign outside of math mode
% Figures and tables can go at their appropriate places in the document rather than at the end. 
% To update bibtex source run ads_importer.py omegaCen_mnras_2015 after latex, then run latex, bibtex, latex, latex (ads_importer.py is available from VS's github, rely's on ADS style refs).


\section{Introduction}
\label{sec:intro}

The Carnegie RR Lyrae Program (CRRP) is a Warm {\it Spitzer} program \citep[][PID 90002]{2012sptz.prop90002F} with the aim of calibrating the mid--infrared (mid--IR) RR Lyrae period--luminosity (PL) relation. Similar to the Carnegie Hubble Program (CHP) \citep{2011AJ....142..192F}, which used mid--IR observations of Cepheids to measure the Hubble constant \citep[$H_{0}$][]{2012ApJ...758...24F}, the results of the CRRP will be used to provide an independent, population II calibration of the extragalactic distance scale, and hence an independent measurement of $H_{0}$. 

In recent years it has become increasingly important to obtain independent direct measurements of \ho. The results of \citet{2011ApJ...730..119R} and \citet{2012ApJ...758...24F}, both which use Cepheids and type Ia supernovae (SNe) as their base, agree very well at $74.4\pm 2.5$~km~s$^{-1}$~Mpc$^{-1}$ and $74.3\pm2.6$~km~s$^{-1}$~Mpc$^{-1}$, respectivly. However, when we consider the latest results from {\it Planck}, who find $67.48\pm0.98$~km~s$^{-1}$~Mpc$^{-1}$ \citep{2015arXiv150201589P}, there is tension. The {\it Planck} study derives their measurement from a model of the cosmic microwave background (CMB), so is completely independent of the \citeauthor{2011ApJ...730..119R} and \citeauthor{2012ApJ...758...24F} results. 

There have been several recent works that have investigated possible sources of uncertainty in the distance ladder that may contribute to the discrepancy between \ho measurements. For example, \citet{2015ApJ...802...20R} examine the differences in star formation rates in type Ia SNe host galaxies. They find that the intrinsic brightness of a SNe Ia may be affected by the local host environment; i.e. whether the SN occurs in a locally star forming or locally passive environment. \citet{2014MNRAS.440.1138E} reanalysed the Cepheid data from \citet{2011ApJ...730..119R}, and found that different outlier rejection criteria lowered the resultant value of \ho to $70.6 \pm 3.3$~km~s$^{-1}$~Mpc$^{-1}$, making it compatible with the value from {\it Planck}. 

The CRRP assess a systematic that was unreachable in the original CHP --- the intrinsic accuracy of the mid--IR Cepheid standard candle distance scale when compared to the standard ruler distance scale of the CMB and Baryon Acoustic Oscillation (BAO) measurements. With only one ``test candle'' it is impossible to make any assessment of this accuracy. However, when we have two standard candles with similar precision we can make meaningful comparisons and assess their systematic accuracy.

RR Lyrae variables are intrinsically fainter than Cepheids, and in the optical follow a much shallower, even horizontal, PL relation \citep{2004ApJS..154..633C}. Determining an accurate distance to an RRL in the $V$ band requires knowledge of its metallicity. However, in more recent years near-- and mid--IR observations have shown the true power of RRL as precision distance indicators. \textit{HST} parallaxes were obtained for several Galactic RRL calibrators \citep{2011AJ....142..187B} and several groups have been studying the populations of RRL in globular clusters and nearby dwarf spheroidal galaxies \citep[e.g.][and references therein]{2013ApJ...767...62G, 2014ApJ...786..147O, 2015ApJ...806..200C, 2015A&A...578A.128K}

Moving to the mid--infrared is well known to minimise the effects of extinction, where $A_{[3.6]}$ and $A_{[4.5]}$ are 16 to 20 times lower than $A_V$ \citet{1989ApJ...345..245C, 2005ApJ...619..931I}. Additionally, the precision of distances obtained from the RRL PL relation is increased. At the wavelengths observed by Warm \textit{Spitzer} (3.6 and 4.5~$\mu$m) we do not see photospheric effects, but only the effects of temperature driving the pulsation; essentially, the mid--infrared light curve is tracing the change in radius of the star over a pulsation cycle. A by--product of this effect is that the intrinsic width of the RRL PL relation is also minimised in the mid--infrared (mid--IR) The PL relation for pulsational variables can be thought of as a two--dimensional projection of the three--dimensional period--luminosity--colour relation (see figure 3 of \citet{1991PASP..103..933M} for a graphical representation). As the colour--width decreases in the mid--IR, the width of the PL naturally decreases. As one moves from the optical to the mid--IR, the slope of the PL relation steepens and its dispersion dramatically decreases, and the slope should asymptotically approach the predicted slope of the period--radius relation, resulting in a slope between $-2.4$ and $-2.8$ \citep{2013ApJ...776..135M}. Through this decrease in dispersion we have found that the intrinsic width of the mid--IR PL for RRL is in fact smaller than for Cepheids -- 0.05~mag compared to 0.10~mag \citep[Monson et al. 2015, in prep,][]{2015arXiv150507858N}. This translates to an uncertainty on an individual RR Lyrae star of 2\%, compared to 4\% for Cepheids. 

In this work we focus on the effect of metallicity on the RR Lyrae (RRL) PL relation. Several Galactic Globular Clusters are being observed as part of CRRP, but $\omega$~Cen is unique in that it exhibits a measureable spread in metallicity \citep{1975ApJ...201L..71F, 2007ApJ...663..296V, 2014ApJ...791..107V}.

There are very few metallic or molecular transition lines in the mid--IR at typical RRL temperatures, so the effects of metallicity on luminosity should be minimised. However, $\omega$~Cen provides the ideal test bed for any effect that we may not have predicted. Such an effect is not out of the realm of possibility; for example, the strength of the CO band head at 4.5~$\mu$m has been found to have a significant dependence on metallicity, and has such prevented the IRAC 4.5~$\mu$m Cepheid observations from being used for distance measurements in the CHP \citep{2011ApJ...743...76S, 2012ApJ...759..146M, 2015arXiv150206995S}. As our concern in this program is systematic precision, we must ensure that similar effects do not plague the RRL distance scale.  

The paper is set out as follows: Section~\ref{sec:observations} details the observations and data reduction. Section~\ref{sec:results} presents the photometry of the $\omega$~Cen RRL. Section~\ref{sec:pl_relation} describes the mid--IR PL relations and Section~\ref{sec:distance_moduli} discusses the application of these to a distance measurement of  $\omega$~Cen. Section~\ref{sec:metallicity} and Section~\ref{sec:discussion} examine the effect of metallicity on mid--IR observations of RRLs and its implications for distance measurements and the extragalactic distance scale. In Section~\ref{sec:conclusions} we present our conclusions.

\section{Observations \& Data Reduction}
\label{sec:observations}
This work combines mid--IR observations from the Warm {\it Spitzer} mission, with supporting near--IR observations from the FourStar instrument on the Baade-Magellan telescope at Las Campanas Observatory \citep{2013PASP..125..654P}. Figure~\ref{fig:omegaCen_fields} shows a $K_s$ FourStar image with the {\it Spitzer} fields outlined, and the positions of known RRL plotted as circles.

\begin{figure}
\begin{center}
\includegraphics[width=80mm]{reworked_fitting_code/final_plots/omegacen_coverage_map_new.pdf}
\caption{A $K_s$--band image of $\omega$~Cen from the FourStar camera, overlaid with a catalog of RRL from \citet{2004A&A...424.1101K} and footprints of the {\it Spitzer} IRAC fields.}
\label{fig:omegaCen_fields}
\end{center}
\end{figure}

\subsection{Warm {\it Spitzer} Data}
\label{sec:spitzer_reduction}
The Warm~\textit{Spitzer} observations for this work were taken as part of the Carnegie RR Lyrae Program. Three fields in $\omega$~Cen were chosen; their positions and the positions of known $\omega$~Cen RRLs are shown in Figure~\ref{fig:omegaCen_fields}. To obtain optimal RRL light curves we observed each field 12 times over approximately 16 hours, roughly corresponding to the period of the longest period RRL we expected in the field. The observations of all three fields were taken on 2013-05-10 and 2013-05-11. Each field was observed using {\it Spitzer} IRAC \citep{2004ApJS..154...10F} with a 30s frame time with a medium scale, gaussian 5--point dither pattern to mitigate any image artefacts. Images were collected in both the 3.6 and 4.5~$\mu$m channels. 
%% VS added info about field shapes and missing colour information.
The elongated field shapes come from the design of IRAC; while the [3.6] channel is collecting on--target data, the [4.5] channel collects off target data ``for free'', and vice versa. We chose to include these off--target fields to maximise the number of RRL in our final sample and to increase the legacy value of our data set to the community. 

The science images were created using \textsc{mopex} \citep{2006SPIE.6274E..0CM}, first running overlap correction on the basic calibrated data (cBCDs) then mosaicking them at 0.6 arcsec pixel scale using the drizzle algorithm. Mosaicked location--correction images were created at the same time. 

%% VS updated the photometry procedure text
PSF photometry was performed using {\sc daophot} and {\sc allframe} \citep{1987PASP...99..191S, 1994PASP..106..250S}. The PSF model was created for each field/filter combination using the first epoch data. This was then applied to each other epoch. As the observations were taken temporally close together the effects of telescope rotation between epochs on the mosaicked PSF were minimal, so making a single good PSF model for each field/filter combination was much more efficient than creating one for every epoch. 

Master star lists for {\sc allframe} were created for each filter/field combination using a median mosaicked image created by {\sc mopex}. We did not use the same single master star list for both filters as only a small proportion ($1/3$) of the 3.6~$\mu$m and 4.5~$\mu$m fields overlap each other. Instead we performed separate {\sc allframe} reductions for each filter, and combined the results after the fact using {\sc daomatch} and {\sc daomaster}. Our mid--IR photometry is calibrated to the standard system set by \citet{2005PASP..117..978R}.

\subsection{FourStar Data}
\label{sec:fourstar_reduction}

$J$, $H$ and $K_s$ data were taken with the FourStar instrument on the Baade-Magellan telescope at Las Campanas Observatory \citep{2013PASP..125..654P} on the nights of June 25, 27 and 28 (2013).  Four epochs were obtained each night in each filter for a total of 12 epochs.  A mosaic of $5\times3$ (slightly overlapping) pointings (tiles) covered a $50\times30$ arcminute field of view centered on $\omega$~Cen.  Each tile consists of a 5 point dither pattern with a 5.8 second exposure time.  Stacked mosaics of the entire field were made as well as individual tiles using a customized pipeline for FourStar data.  The purpose of the individual tiles is to provide photometry with better time resolution than the large mosaic.  

PSF photometry of the tiles was performed using \textsc{daophot} and \textsc{allframe} \citep{1987PASP...99..191S, 1994PASP..106..250S}. A PSF model was created for each epoch/tile/filter combination.  A master star list for \textsc{allframe} was created from the final $K_s$ mosaic and the multi-wavelength/epoch results were combined using \textsc{daomatch} and \textsc{daomaster}.  Our final photometry is calibrated to the 2MASS standard system \citep{2006AJ....131.1163S}. 

\subsection{Crowding}
\label{sec:crowding}

The primary limiting factor in the data is crowding: 77 RRLs out of the original catalog of 192 \citep{2004A&A...424.1101K} were rejected due to crowding. We compared the {\it Spitzer} images to the FourStar $K_s$--band image. The 0.159 arcsec/pixel resolution of the $K_s$ band image enabled us to better see which stars were significantly contaminated. Our full, uncrowded RRL sample consists of 97 stars in $J$ and $H$, 99 in $K_s$, 37 in 3.6~$\mu$m, and 43 in 4.5~$\mu$m.

\section{Results}
\label{sec:results}

Our final photometry catalog, including magnitudes and errors for $J\!H\!K_s$, 3.6~$\mu$m, and 4.5~$\mu$m is presented in Table~\ref{tab:phot}.
%% See comments in appendix for how I got the appendix table numbering to work properly.
The average magnitudes presented in Table~\ref{tab:phot} are flux averages, and the photometric uncertainties of the time series data are the error on the mean.
% VS rewrote this paragraph to be mor concise. It says exactly the same thing as you wrote. 

%The average magnitudes are calculated by first converting the individual values into fluxes, then taking the mean flux and converting that back to a magnitude. The photometric errors of the time series data are added in quadrature and divided by the square root of the total number of observations to obtain the final error value.

\section{Period--Luminosity Relations}
\label{sec:pl_relation}

%% VS: I'm doing some rearranging here. All the information is here but I think it needs a bit of reorganising to flow better. 
%% I'm trying to get it to the point where you have all the relavent information in order:
%% PL definitions
%% Description of fitting method
%% Metallicitiy definitions (essentially more data)
%% Full sample PL fits
%% Cut sample PL fits (can't describe cut sample until you know what you're cutting from)
%% Also moving all references to the distance modulus plots to the next section. Trying to focus on on result at a time. 
%% Important to remember (and the editor will pull you up on this) all figures/tables must be in the paper in the order in which they are referred to (so fig 1 *must* be referred to first, fig 2 second etc) and you can't have any figures that you don't refer to in the text. I'm not going to fuss with this too much until you've got the new figures in. 

%% Definition of PL relations and fitting method
We test both empirical and theoretical parameters as fiducial in our PL fitting. We use the theoretical near-- and mid--infrared PL relation parameters presented in \citet{2015ApJ...808...50M} and Braga et al. (in prep.), along with the empirical PL relation parameters derived from photometry of RRLs in the globular cluster M4 (NGC 6121) from \citet{2015arXiv150507858N} for comparison. With the use of preexisting PL relation coefficients, the distance modulus becomes the only free parameter in our fit. We fit all distance moduli using an unweighted least--squares method, and fit the distance modulus to each pulsation mode in each wavelength separately.

%% PL equations

The theoretical RRL PL relations are described in Table~\ref{tab:pl_table_theo}. The relations take the form
\begin{equation}M = a + b\times\log P + c\times[\text{Fe/H}]\end{equation}
where $a$, $b$, and $c$ are theoretically derived coefficients.
\begin{table}
\centering
\caption{Theoretical near--IR RRL period--luminosity relation coefficients for $\omega$ Cen \citep{2015ApJ...808...50M}, for relations of the form $M = a + b \times \log P + c \times [\text{Fe/H}]$ with scatter $\sigma$.}
\label{tab:pl_table_theo}
\begin{tabular}{l||c|c|c|c|r} 
\hline \hline
Band & Mode & $a$   & $b$   & $c$   & $\sigma$ \\
\hline
$J$ & RRab & $-0.510$ & $-1.980$ & $0.170$ & $0.060$ \\
       & RRc & $-1.070$ & $-2.460$ & $0.150$ & $0.040$ \\
$H$ & RRab & $-0.760$ & $-2.240$ & $0.190$ & $0.040$\\
       & RRc & $-1.310$ & $-2.700$ & $0.160$ & $0.020$\\
$K_s$ & RRab & $-0.820$ & $-2.270$ & $0.180$ & $0.030$\\
           & RRc & $-1.370$ & $-2.720$ & $0.150$ & $0.020$ \\       
% $[3.6]$ & RRc & $-1.344$ & $-2.718$ & $0.152$ & $0.021$ \\
%             & RRab & $-0.786$ & $-2.276$ & $0.184$ & $0.035$ \\
% $[4.5]$ & RRc & $-1.348$ & $-2.720$ & $0.153$ & $0.021$ \\         
%             & RRab & $-0.775$ & $-2.262$ & $0.190$ & $0.036$ \\
            \hline
\end{tabular}
\end{table}


We also use the the empirical mid--IR RRL PL relations from \citet{2015arXiv150507858N} for comparison. These are described in Table~\ref{tab:pl_table_m4}. The relations take the form
\begin{equation}M = a + b \times (\log (P) + P_0) \end{equation}
where $a$ and $b$ are empirically derived coefficients and $P_0$ is the absolute value of the mean period.

\begin{table}
\centering
\caption{Empirical mid--IR RRL period--luminosity relation coefficients for $\omega$ Cen \citep{2015arXiv150507858N}, for relations of the form $M = a + b \times (\log (P) + P_0)$ with scatter $\sigma$.} 
\label{tab:pl_table_m4}
\begin{tabular}{l||c|c|c|c|c|r} 
\hline \hline
Band & Mode & $a$ & $b$ & $P_0$ & $\sigma$ \\
\hline
$[3.6]$ & RRab & $-0.558$ & $-2.370$ & $0.260$ & $0.035$ \\
            & RRc & $-0.192$ & $-2.658$ & $0.550$ & $0.021$ \\
$[4.5]$ & RRab & $-0.593$ & $-2.355$ & $0.260$ & $0.036$ \\
            & RRc & $-0.240$ & $-2.979$ & $0.550$ & $0.021$ \\ 
            \hline
\end{tabular}
\end{table}

%% Metallicity description

The theoretical PL relations for the near--IR have a metallicity-dependent term; however, we do not have known metallicities for all stars in our sample. We therefore use a single [Fe/H] value that is the average [Fe/H] of the RRLs for which there are known metallicities. Using spectroscopic metallicities from \citet{2006ApJ...640L..43S}, we obtain an average [Fe/H] of $-1.567$. This will be discussed further in section~\ref{sec:metallicity}.

% The theoretical PL relations for both the near-- and mid--IR have a metallicity-dependent term. 
% Not all stars in our sample have known metallicity values, so we use an average [Fe/H] value for all RRLs in the cluster. We use both photometric \citep{2000AJ....119.1824R} and spectroscopic \citep{2006ApJ...640L..43S} metallicities for comparison. We find that there is little correspondence between individual metallicity measurements for stars which have both spectroscopic and photometric metallicity values, as shown in Figure~\ref{fig:metallicity_comparison}, and that the average metallicities of the spectroscopic and photometric catalogs differ by nearly 0.1 dex. We use a mean photometric [Fe/H] of  $-1.584$ and a spectroscopic [Fe/H] of $-1.677$.

%% Start by saying what the big sample is:

Our full, uncrowded RRL sample consists of 96 stars in $J$ and $H$, 98 in $K_s$, 36 in 3.6~$\mu$m, and 43 in 4.5~$\mu$m. The PL fits to this sample are shown in Figure~\ref{fig:omegaCen_pl_m4}.

\begin{figure}
\begin{center}
\includegraphics[width=80mm]{reworked_fitting_code/final_plots/multiwavelength_PL_m4_clipped.pdf}
\caption{PL relations for $J\!H\!K_s$, 3.6~$\mu$m, and 4.5~$\mu$m photometry assuming an [Fe/H]$=-1.627$, corresponding to the average spectroscopic metallicity for stars in our sample from \citep{2006ApJ...640L..43S}. All uncrowded RRL for which we have photometry are included in these fits.}
\label{fig:omegaCen_pl_m4}
\end{center}
\end{figure}

For the final PL relations, we use only the stars for which we have photometry in all five bandpasses. Our final RRL sample consists of 25 RRL (12 fundamental mode and 13 first overtone). This sample reduces bias by ensuring that the same range of periods and metallicities are sampled for each wavelength.

\begin{comment}
\begin{figure}
\begin{center}
\includegraphics[width=80mm]{reworked_fitting_code/final_plots/multiwavelength_PL_samestars_phot_unweighted.pdf}
\caption{PL relations for $J\!H\!K_s$, 3.6~$\mu$m, and 4.5~$\mu$m photometry assuming an [Fe/H]$=-1.584$, corresponding to the average photometric metallicity from \citet{2000AJ....119.1824R}. Only those RRL that appear in all five near-- and mid--infrared bands are included in the fit.}
\label{fig:omegaCen_pl_phot}
\end{center}
\end{figure}

\begin{figure}
\begin{center}
\includegraphics[width=80mm]{reworked_fitting_code/final_plots/multiwavelength_PL_samestars_spect_unweighted.pdf}
\caption{PL relations for $J\!H\!K_s$, 3.6~$\mu$m, and 4.5~$\mu$m photometry assuming an [Fe/H]$=-1.677$, corresponding to the average spectroscopic metallicity from \citet{2006ApJ...640L..43S}.  Only those RRL that appear in all five near-- and mid--infrared bands are included in the fit.}
\label{fig:omegaCen_pl_spect}
\end{center}
\end{figure}
\end{comment}

\section{Distance Moduli}
\label{sec:distance_moduli}

We combine the uncorrected distance moduli from each bandpass to obtain a mean reddening--corrected distance modulus. We fit the near--infrared reddening law from \citet{1989ApJ...345..245C} and mid--infrared law from \citet{2005ApJ...619..931I} simultaneously, assuming the ratio of total to selective absorption $R_V = 3.1$. (We tested a value of $R_V = 3.23$ as well, and it changed the results by less than 1\%.) The resulting fits are shown in Figures~\ref{fig:omegaCen_dist_phot} and \ref{fig:omegaCen_dist_spect}.

% put some dank memes here

\begin{figure}
\begin{center}
\includegraphics[width=80mm]{reworked_fitting_code/final_plots/multiwavelength_distance_m4_clipped_mean.pdf}
\caption{Distance moduli for the final sample of $J\!H\!K_s$, 3.6~$\mu$m, and 4.5~$\mu$m photometry. The filled circles are the mean distance moduli using both RRab and RRc stars, the unfilled circles are the distance moduli using only RRab stars, and the filled triangles are distance moduli using only RRc stars.}
\label{fig:omegaCen_dist_m4_mean}
\end{center}
\end{figure}

\begin{figure}
\begin{center}
\includegraphics[width=80mm]{reworked_fitting_code/final_plots/multiwavelength_distance_m4_clipped_ab.pdf}
\caption{Distance moduli for the final sample of $J\!H\!K_s$, 3.6~$\mu$m, and 4.5~$\mu$m photometry}
\label{fig:omegaCen_dist_m4_ab}
\end{center}
\end{figure}

Using the empirical PL relations for 3.6~$\mu$m and 4.5~$\mu$m and the theoretical PL relations for $J\!H\!K_s$, we derive a true dereddened distance modulus of $\mu_0 = 13.777 \pm 0.011$ and an $E(B-V)$ of $0.075 \pm $(something), as shown in Figure~\ref{fig:omegaCen_dist_m4_mean}.

The anomalously high distance modulus for 4.5~$\mu$m in Figure~\ref{fig:omegaCen_dist_m4_mean} is caused exclusively by the first overtone pulsators, which have the lowest signal-to-noise measurements and are therefore most vulnerable to crowding effects. If we remove the first overtone pulsators from the 4.5~$\mu$m data, the derived distance modulus for 4.5~$\mu$m decreases substantially, as shown in Figure~\ref{fig:omegaCen_dist_m4_ab}, resulting in an excellent fit of all points to the reddening curve. From these distance moduli we derive a true dereddened distance modulus of \textbf{something} %$\mu_0 = 13.770 \pm 0.005$ and $E(B-V) = 0.084 \pm$(something).

\begin{comment}
\begin{figure}
\begin{center}
\includegraphics[width=80mm]{reworked_fitting_code/final_plots/period_color_clipped.pdf}
\caption{Period color thing}
\label{fig:period_color}
\end{center}
\end{figure}
\end{comment}

\section{Metallicity}
\label{sec:metallicity}

$\omega$ Cen is ideal for examining the RRL period--luminosity--metallicity relation, [metallicity spread]%
A metallicity spread this wide is not found in any other Galactic globular cluster. One of the advantages of using globular clusters to calibrate PL coefficients is that all stars in a cluster can be considered to be at the same distance from Earth. We can therefore assume that any dispersion in the PL relation is a combination of the a) the intrinsic dispersion of the PL relation, b) the photometric uncertainties, and c) dispersion induced by the spread in metallicity of the RRL. Since we have measured the intrinsic dispersion of the RRL PL from the cluster M4, \citet{2015arXiv150507858N}) and our photometric uncertainties are well understood, so the only unknown in this problem is the dispersion due to the spread in metallicity of the cluster. 

$\omega$~Cen is unique in that we can also take a second approach to establishing the metallicity effect on the RRL PL relation. As it is such an interesting system, $\omega$~Cen is extremely well studied and many of its RRL have spectroscopic or photometric metallicities in the literature \citep[e.g.][]{2006ApJ...640L..43S, 2000AJ....119.1824R}. As another test of the effect of metallicity, we use these measurements to assess the $\gamma$ parameter for $\omega$~Cen, where 
\begin{equation} \label{eqn:gamma}
\gamma = \dfrac {\Delta \text{mag}} {[\text{Fe/H}]}\text{,}
\end{equation}

similar to $\gamma$ used to quantify the effect of metallicity on the zero--point of the Cepheid PL relation \citep{1998ApJ...498..181K}. 

Theoretical models suggest that the metallicity dependence of the RRL PL relation should decrease monotonically from the optical to the near-infrared \citep{2001MNRAS.326.1183B, 2004ApJS..154..633C}. Observational evidence corroborates this; previous investigations performed on WISE data suggest no obvious metallicity dependence in the mid-IR PL relations \citep{2013ApJ...776..135M}.

In the case of Cepheids, \citet{2011ApJ...743...76S} and \citet{2015arXiv150206995S} have shown that in the 4.5~$\mu$m bandpass there is absorption due to a CO bandhead at 4.65~$\mu$m, which strengthens the metallicity dependence of the PL relation in this bandpass. However, this effect is due to the low temperature of Cepheid atmospheres and disappears in the hottest, shortest-period Cepheids, as the CO dissociates at temperatures above 6000 K \citep{2012ApJ...759..146M}. As even the coolest RRL have temperatures over 6000 K \citep{1971PASP...83..697I}, we expect to see no such CO absorption in the 4.5~$\mu$m PL relation. If there are any other unanticipated metallicity effects, they must be smaller than the dispersion of the PL relations themselves.

If there is any correlation between [Fe/H] and the PL residuals, we expect it to be a linear one, consistent with the theoretical metallicity terms in the PL relation, $c\times[\text{Fe/H}]$; we fit a relation of the form
\begin{equation}
\Delta\text{mag} = \gamma \times[\text{Fe/H}] + d
\end{equation}
to the 3.6~$\mu$m and 4.5~$\mu$m PL residuals and metallicity values for stars with known individual metallicity values, as shown in Figure~\ref{fig:metallicity_residuals}. We find that although the scatter in the 3.6~$\mu$m and 4.5~$\mu$m PL relations is higher for $\omega$~Cen than it is for M4 \citep{2015arXiv150507858N, 2015ApJ...799..165B}, there is no evidence that it is due to metallicity. When we examine [Fe/H] vs. $\Delta$3.6~$\mu$m and $\Delta$4.5~$\mu$m, $\gamma$ is within $1\sigma$ of zero for all fits, indicating that there is no significant metallicity dependence in the PL residuals.

\begin{figure}
\begin{center}
\includegraphics[width=80mm]{reworked_fitting_code/final_plots/metallicity_comparison_all_clipped.pdf}
\caption{Spectroscopic vs. photometric measurements of [Fe/H] for RRLs in $\omega$~Cen.}
\label{fig:metallicity_comparison}
\end{center}
\end{figure}


\begin{figure}
\begin{center}
\includegraphics[width=80mm]{reworked_fitting_code/final_plots/metallicity_vs_residuals_m4_clipped.pdf}
\caption{Photometric and spectroscopic [Fe/H] values vs. period--luminosity residuals in 3.6~$\mu$m and 4.5~$\mu$m, with the $\gamma$ parameter from equation 4 in the top right corner.}
\label{fig:metallicity_residuals}
\end{center}
\end{figure}


\section{Discussion}
\label{sec:discussion}

%Alternative explanations for the increased scatter relative to M4

%The metallicity terms in the theoretical mid--infrared relations are comparable to the metallicity terms in the near--infrared (see Table~\ref{tab:pl_table_theo}), which is counter to previous predictions that the metallicity dependence should decrease with increasing wavelength \citep{2004ApJS..154..633C, 2001MNRAS.326.1183B}. This may be one reason for the $\sim\!\!0.1$~mag discrepancy between the distance moduli for 3.6~$\mu$m and 4.5~$\mu$m derived using the theoretical parameters vs. the empirical parameters.

%Our investigation into the metallicity effects in the mid--infrared is limited primarily by the quality of the metallicity measurements of individual stars. {\bf Did someone say Eric Persson was working on this at some point...? Should I say we can expect a better dataset at some point?}

\section{Conclusions}
\label{sec:conclusions}

\section*{Acknowledgements}
\label{sec:acknowledgements}

We thank Eric Persson for his many contributions to this project.

This work is based on observations made with the Spitzer Space Telescope, which is operated by the Jet Propulsion Laboratory, California Institute of Technology under a contract with NASA. Support for this work was provided by NASA through an award issued by JPL/Caltech.

This publication makes use of data products from the Two Micron All Sky Survey, which is a joint project of the University of Massachusetts and the Infrared Processing and Analysis Center/California Institute of Technology, funded by the National Aeronautics and Space Administration and the National Science Foundation.

This research has made use of the NASA/IPAC Extragalactic Database (NED), which is operated by the Jet Propulsion Laboratory, California Institute of Technology, under contract with the National Aeronautics and Space Administration.


%%%%%%%%%%%%%%%%%%%%%%%%%%%%%%%%%%%%%%%%%%%%%%%%%%

%%%%%%%%%%%%%%%%%%%% REFERENCES %%%%%%%%%%%%%%%%%%

% The best way to enter references is to use BibTeX:

\bibliographystyle{mnras}
\bibliography{omegaCen_mnras_2015}
 % if your bibtex file is called example.bib


% Alternatively you could enter them by hand, like this:
% This method is tedious and prone to error if you have lots of references

%%%%%%%%%%%%%%%%%%%%%%%%%%%%%%%%%%%%%%%%%%%%%%%%%%

%%%%%%%%%%%%%%%%% APPENDICES %%%%%%%%%%%%%%%%%%%%%

\clearpage
\newpage

\appendix

%% To get the appendix table numbering right you need to have a section after the appendix command. This way the table will be referred to as Table A1, rather than Table 1 (which you already have another one of in the main text).

\section{Appendix: Photometry of RRLs}
\label{sec:phot_table_appendix}
\onecolumn
\begin{landscape}
\begin{center}
\scriptsize{
\begin{longtable}{lcccccccccccccccccccr}
\caption{$J\!H\!K_s$, 3.6~$\mu$m, and 4.5~$\mu$m photometry of the RRLs in $\omega$~Cen\label{tab:phot}} 
\tabularnewline 
ID & RA (J2000) & Dec (J2000) & Mode & $P$ (days) & $J$  & $\sigma_{J}$ & $H$  & $\sigma_{H}$ & $K_s$  & $\sigma_{K_s}$ & [3.6] & $\sigma_{{[3.6]}}$ & $\Delta [3.6]$ & [4.5] & $\sigma_{{[4.5]}}$ & $\Delta [4.5]$ & [Fe/H], p   & $\sigma_{[\text{Fe/H}]}$, p & [Fe/H], s & $\sigma_{[\text{Fe/H}]}$, s \\
\hline
\endfirsthead
\multicolumn{4}{c}%
{\tablename\ \thetable\ -- \textit{Continued from previous page}} \\
\hline 
ID & RA (J2000) & Dec (J2000) & Mode & $P$ (days) & $J$  & $\sigma_{J}$ & $H$  & $\sigma_{H}$ & $K_s$  & $\sigma_{K_s}$ & [3.6] & $\sigma_{{[3.6]}}$ & $\Delta [3.6]$ & [4.5] & $\sigma_{{[4.5]}}$ & $\Delta [4.5]$ & [Fe/H], p   & $\sigma_{[\text{Fe/H}]}$, p & [Fe/H], s & $\sigma_{[\text{Fe/H}]}$, s \\
\hline
\endhead
\hline \multicolumn{4}{r}{\textit{Continued on next page}} \\
\endfoot
\hline
\endlastfoot
3&13:25:56.15&-47:25:53.8&RRab&0.841&13.247&0.017&12.982&0.018&12.882&0.017&12.841&0.039&-0.086&12.708&0.036&0.031&-1.540&0.050&---&--- \\
4&13:26:12.93&-47:24:18.8&RRab&0.627&13.475&0.016&13.219&0.021&13.133&0.020&13.030&0.036&0.027&13.026&0.035&0.013&-1.740&0.050&---&--- \\
5&13:26:18.33&-47:23:12.4&RRab&0.515&13.700&0.017&13.549&0.020&13.507&0.027&13.387&0.043&-0.128&13.340&0.030&-0.100&-1.350&0.080&-1.240&0.110 \\
7&13:27:00.90&-47:14:00.5&RRab&0.713&13.333&0.009&13.151&0.031&13.036&0.018&---&---&---&---&---&---&-1.460&0.080&---&--- \\
8&13:27:48.45&-47:28:20.3&RRab&0.521&13.505&0.015&13.258&0.017&13.223&0.014&---&---&---&---&---&---&-1.910&0.280&---&--- \\
9&13:25:59.58&-47:26:24.0&RRab&0.523&13.776&0.017&13.534&0.021&13.470&0.016&13.315&0.036&-0.071&13.279&0.039&-0.055&-1.490&0.060&---&--- \\
11&13:26:30.59&-47:23:01.6&RRab&0.565&13.481&0.014&13.307&0.028&13.219&0.025&13.050&0.058&---&---&---&---&-1.670&0.130&-1.610&0.220 \\
13&13:25:58.18&-47:25:21.6&RRab&0.669&13.353&0.019&13.081&0.022&13.058&0.017&12.918&0.032&0.073&12.860&0.031&0.114&-1.910&0.000&---&--- \\
14&13:25:59.74&-47:39:09.6&RRc&0.377&13.588&0.011&13.343&0.020&13.365&0.016&---&---&---&13.299&0.045&---&-1.710&0.130&---&--- \\
15&13:26:27.11&-47:24:38.0&RRab&0.811&13.245&0.018&13.020&0.031&12.954&0.025&13.149&0.084&---&---&---&---&-1.640&0.390&-1.680&0.180 \\
16&13:27:37.69&-47:37:34.8&RRc&0.330&13.680&0.015&13.502&0.022&13.437&0.018&---&---&---&---&---&---&-1.290&0.080&-1.650&0.460 \\
18&13:27:45.11&-47:24:56.6&RRab&0.622&13.371&0.010&13.131&0.024&13.100&0.016&13.006&0.043&---&---&---&---&-1.780&0.280&---&--- \\
20&13:27:14.05&-47:28:06.3&RRab&0.616&13.410&0.015&13.210&0.036&13.125&0.025&13.060&0.039&0.017&12.940&0.029&0.119&---&---&-1.520&0.340 \\
21&13:26:11.17&-47:25:58.8&RRc&0.381&13.578&0.016&13.399&0.027&13.361&0.020&13.301&0.047&-0.003&13.200&0.032&0.061&-0.900&0.110&---&--- \\
23&13:26:46.50&-47:24:39.5&RRab&0.511&13.941&0.025&13.794&0.048&13.658&0.033&13.325&0.064&---&---&---&---&-1.080&0.140&-1.350&0.580 \\
30&13:26:15.94&-47:29:56.0&RRc&0.404&13.521&0.021&13.287&0.046&13.251&0.030&13.188&0.047&0.041&13.071&0.060&0.112&-1.750&0.170&-1.620&0.280 \\
32&13:27:03.32&-47:21:38.9&RRab&0.620&13.508&0.009&13.244&0.018&13.132&0.018&---&---&---&---&---&---&-1.530&0.160&---&--- \\
33&13:25:51.60&-47:29:05.8&RRab&0.602&13.338&0.015&13.106&0.022&13.091&0.019&---&---&---&13.006&0.035&---&-2.090&0.230&-1.580&0.420 \\
34&13:26:07.21&-47:33:10.4&RRab&0.734&13.273&0.014&13.018&0.014&12.916&0.013&---&---&---&12.838&0.065&---&-1.710&0.000&---&--- \\
35&13:26:53.21&-47:22:34.7&RRc&0.387&13.586&0.012&13.463&0.024&13.356&0.023&---&---&---&---&---&---&-1.560&0.080&-1.630&0.360 \\
36&13:27:10.11&-47:15:29.8&RRc&0.380&13.534&0.007&13.372&0.019&13.307&0.014&---&---&---&---&---&---&-1.490&0.230&---&--- \\
38&13:27:03.30&-47:36:30.2&RRab&0.779&13.226&0.015&12.943&0.019&12.814&0.018&---&---&---&---&---&---&-1.750&0.180&-1.640&0.400 \\
39&13:27:59.77&-47:34:42.3&RRc&0.393&13.560&0.009&13.415&0.014&13.308&0.014&---&---&---&---&---&---&-1.960&0.290&---&--- \\
40&13:26:24.56&-47:30:46.2&RRab&0.634&13.517&0.022&13.250&0.051&13.153&0.033&13.062&0.049&-0.016&13.416&0.056&-0.388&-1.600&0.080&-1.620&0.190 \\
44&13:26:22.39&-47:34:35.3&RRab&0.568&13.677&0.014&13.425&0.023&13.368&0.018&---&---&---&13.132&0.036&---&-1.400&0.120&-1.290&0.350 \\
45&13:25:30.88&-47:27:21.0&RRab&0.589&13.513&0.015&13.201&0.015&13.164&0.014&---&---&---&13.070&0.028&---&-1.780&0.250&---&--- \\
46&13:25:30.23&-47:25:51.8&RRab&0.687&13.299&0.016&12.998&0.017&12.947&0.014&---&---&---&---&---&---&-1.880&0.170&---&--- \\
47&13:25:56.46&-47:24:12.0&RRc&0.485&13.420&0.020&13.223&0.018&13.150&0.018&13.099&0.030&-0.080&13.073&0.026&-0.126&-1.580&0.310&---&--- \\
49&13:26:07.78&-47:37:55.5&RRab&0.605&13.566&0.012&13.238&0.019&13.220&0.016&---&---&---&13.099&0.049&---&-1.980&0.110&---&--- \\
51&13:26:42.66&-47:24:21.4&RRab&0.574&13.597&0.014&13.378&0.033&13.270&0.029&13.315&0.083&---&---&---&---&-1.640&0.210&-1.840&0.230 \\
54&13:26:23.54&-47:18:47.7&RRab&0.773&13.281&0.016&12.998&0.017&12.954&0.015&12.799&0.030&---&---&---&---&-1.660&0.120&-1.800&0.230 \\
56&13:25:55.53&-47:37:44.1&RRab&0.568&13.643&0.009&13.386&0.022&13.353&0.017&---&---&---&13.232&0.035&---&-1.260&0.150&---&--- \\
57&13:27:49.38&-47:36:50.5&RRab&0.794&13.234&0.015&12.995&0.018&12.882&0.014&---&---&---&---&---&---&-1.890&0.140&---&--- \\
59&13:26:18.43&-47:29:46.7&RRab&0.519&13.727&0.023&13.424&0.043&13.391&0.033&13.248&0.071&0.005&13.418&0.064&-0.184&-1.000&0.280&---&--- \\
63&13:25:07.96&-47:36:54.1&RRab&0.826&13.223&0.017&12.862&0.017&12.869&0.012&---&---&---&---&---&---&-1.730&0.090&---&--- \\
64&13:26:02.22&-47:36:19.2&RRc&0.344&13.638&0.013&13.438&0.022&13.407&0.022&---&---&---&13.314&0.044&---&-1.460&0.230&---&--- \\
67&13:26:28.62&-47:18:46.9&RRab&0.564&13.610&0.014&13.384&0.016&13.326&0.015&13.368&0.047&---&---&---&---&-1.100&0.000&-1.190&0.230 \\
68&13:26:12.80&-47:19:35.7&RRc&0.535&13.258&0.021&13.004&0.015&12.970&0.015&12.928&0.050&---&---&---&---&-1.600&0.010&---&--- \\
69&13:25:11.02&-47:37:33.5&RRab&0.635&---&---&---&---&13.112&0.014&---&---&---&---&---&---&-1.520&0.140&---&--- \\
70&13:27:27.76&-47:33:42.7&RRc&0.391&13.529&0.013&13.282&0.029&13.254&0.022&---&---&---&---&---&---&-1.940&0.150&-1.740&0.300 \\
72&13:27:33.11&-47:16:22.9&RRc&0.385&13.554&0.010&13.339&0.017&13.311&0.014&---&---&---&---&---&---&-1.320&0.220&---&--- \\
73&13:25:53.75&-47:16:10.8&RRab&0.575&13.480&0.018&13.251&0.017&13.215&0.016&---&---&---&---&---&---&-1.500&0.090&---&--- \\
74&13:27:07.22&-47:17:33.9&RRab&0.503&13.622&0.008&13.457&0.016&13.405&0.015&---&---&---&---&---&---&-1.830&0.360&---&--- \\
75&13:27:19.70&-47:18:46.5&RRc&0.422&13.410&0.011&13.175&0.028&13.137&0.025&---&---&---&---&---&---&-1.490&0.080&-1.820&0.990 \\
76&13:26:57.23&-47:20:07.7&RRc&0.338&13.634&0.012&13.488&0.017&13.449&0.020&---&---&---&---&---&---&-1.450&0.130&---&--- \\
79&13:28:24.99&-47:29:25.2&RRab&0.608&13.382&0.010&13.162&0.016&13.123&0.015&---&---&---&---&---&---&-1.390&0.180&---&--- \\
81&13:27:36.68&-47:24:48.3&RRc&0.389&13.542&0.013&13.326&0.033&13.286&0.025&13.248&0.076&---&---&---&---&-1.720&0.310&-1.990&0.430 \\
82&13:27:35.61&-47:26:30.3&RRc&0.336&13.579&0.016&13.324&0.024&13.296&0.018&---&---&---&13.827&0.104&---&-1.560&0.200&-1.710&0.560 \\
83&13:27:08.42&-47:21:34.1&RRc&0.357&13.603&0.010&13.431&0.024&13.370&0.022&---&---&---&---&---&---&-1.300&0.220&---&--- \\
84&13:24:47.45&-47:29:56.5&RRab&0.580&---&---&12.833&0.017&12.781&0.016&---&---&---&---&---&---&-1.470&0.100&---&--- \\
85&13:25:06.49&-47:23:34.0&RRab&0.743&13.344&0.011&---&---&---&---&---&---&---&---&---&---&-1.870&0.310&---&--- \\
94&13:25:57.06&-47:22:46.1&RRc&0.254&14.070&0.024&13.934&0.022&13.870&0.027&13.858&0.038&-0.092&13.799&0.029&-0.014&-1.000&0.110&---&--- \\
95&13:25:24.95&-47:28:53.2&RRc&0.405&13.497&0.015&13.269&0.017&13.264&0.017&---&---&---&13.178&0.024&---&-1.840&0.550&---&--- \\
97&13:27:08.49&-47:25:30.9&RRab&0.692&13.302&0.010&13.143&0.029&13.034&0.022&12.964&0.061&-0.007&12.702&0.064&0.237&-1.560&0.370&-1.740&0.170 \\
101&13:27:30.24&-47:29:51.0&RRc&0.341&13.708&0.016&13.484&0.030&13.436&0.023&---&---&---&---&---&---&-1.880&0.320&---&--- \\
102&13:27:22.11&-47:30:12.3&RRab&0.691&13.320&0.012&13.033&0.022&12.993&0.020&12.984&0.049&-0.027&13.056&0.072&-0.116&-1.840&0.130&-1.650&0.160 \\
103&13:27:14.29&-47:28:36.3&RRc&0.329&13.620&0.018&13.409&0.040&13.377&0.034&12.960&0.071&---&13.024&0.066&---&-1.920&0.110&-1.780&0.270 \\
104&13:28:07.76&-47:33:44.9&RRab&0.867&13.732&0.096&13.626&0.154&13.452&0.141&---&---&---&---&---&---&-1.830&0.180&---&--- \\
105&13:27:46.02&-47:32:43.9&RRc&0.335&13.768&0.014&13.615&0.020&13.533&0.018&---&---&---&---&---&---&-1.240&0.180&---&--- \\
107&13:27:14.05&-47:30:57.9&RRab&0.514&13.597&0.017&13.340&0.038&13.301&0.030&13.535&0.219&---&13.351&0.076&---&-1.360&0.110&---&--- \\
115&13:26:12.30&-47:34:17.5&RRab&0.630&13.401&0.012&13.176&0.017&13.103&0.013&---&---&---&---&---&---&-1.870&0.010&-1.640&0.320 \\
117&13:26:19.91&-47:29:21.0&RRc&0.422&13.480&0.020&13.274&0.043&13.202&0.031&13.110&0.044&0.071&12.949&0.043&0.179&-1.680&0.250&---&--- \\
120&13:26:25.52&-47:32:48.6&RRab&0.549&13.525&0.049&13.072&0.079&13.135&0.094&12.958&0.066&0.237&12.927&0.055&0.250&-1.390&0.060&-1.150&0.160 \\
121&13:26:28.17&-47:31:50.5&RRc&0.304&13.741&0.016&13.648&0.033&13.531&0.026&13.414&0.037&0.144&13.302&0.033&0.249&-1.460&0.130&-1.830&0.400 \\
122&13:26:30.31&-47:33:02.2&RRab&0.635&13.369&0.018&13.132&0.042&13.062&0.024&13.057&0.052&-0.012&13.019&0.043&0.008&-2.020&0.180&-1.790&0.210 \\
122&13:26:30.31&-47:33:02.2&RRab&0.635&13.369&0.018&13.132&0.042&13.062&0.024&13.057&0.052&-0.012&12.956&0.105&0.071&-2.020&0.180&-1.790&0.210 \\
124&13:26:54.49&-47:39:07.5&RRc&0.332&13.708&0.013&13.510&0.018&13.482&0.023&---&---&---&---&---&---&-1.330&0.230&---&--- \\
125&13:26:48.92&-47:41:03.7&RRab&0.593&13.420&0.015&13.200&0.016&13.153&0.015&---&---&---&---&---&---&-1.670&0.220&-1.810&0.380 \\
126&13:28:08.03&-47:40:46.7&RRc&0.342&13.642&0.011&13.467&0.017&13.370&0.016&---&---&---&---&---&---&-1.310&0.130&---&--- \\
127&13:25:19.36&-47:28:37.6&RRc&0.305&---&---&---&---&13.579&0.018&---&---&---&13.573&0.063&---&-1.590&0.080&---&--- \\
128&13:26:17.75&-47:30:13.0&RRab&0.835&13.207&0.018&12.927&0.032&12.810&0.020&---&---&---&12.445&0.074&---&-1.880&0.040&---&--- \\
130&13:26:09.93&-47:13:40.0&RRab&0.493&13.688&0.021&13.527&0.032&13.418&0.025&---&---&---&---&---&---&-1.460&0.170&---&--- \\
147&13:27:15.86&-47:31:09.2&RRc&0.423&13.397&0.012&12.934&0.041&13.083&0.022&---&---&---&12.585&0.096&---&-1.660&0.140&---&--- \\
149&13:27:32.94&-47:13:43.6&RRab&0.683&13.354&0.015&13.061&0.035&13.024&0.024&---&---&---&---&---&---&-1.210&0.240&---&--- \\
150&13:27:40.21&-47:36:00.1&RRab&0.899&13.068&0.019&12.757&0.025&12.692&0.018&---&---&---&---&---&---&-1.760&0.340&---&--- \\
163&13:25:49.42&-47:20:21.5&RRc&0.313&13.763&0.019&13.557&0.016&13.545&0.025&---&---&---&---&---&---&-1.180&0.270&---&--- \\
168&13:25:52.78&-47:32:02.9&RRc&0.321&14.176&0.015&14.000&0.020&13.960&0.018&---&---&---&---&---&---&---&---&---&--- \\
169&13:27:20.47&-47:23:59.1&RRc&0.319&13.805&0.013&13.735&0.019&13.652&0.025&13.734&0.050&-0.232&14.001&0.116&-0.512&---&---&-1.650&0.190 \\
184&13:27:28.50&-47:31:35.4&RRc&0.303&13.778&0.012&13.624&0.028&13.536&0.019&---&---&---&---&---&---&---&---&---&--- \\
185&13:26:04.13&-47:21:45.0&RRc&0.333&13.701&0.016&13.545&0.018&13.508&0.023&13.496&0.036&-0.043&13.479&0.033&-0.046&---&---&---&--- \\
261&13:27:15.41&-47:21:29.5&RRc&0.403&13.431&0.009&13.212&0.019&13.113&0.020&---&---&---&---&---&---&---&---&-1.500&0.350 \\
263&13:26:13.13&-47:26:09.7&RRab&1.012&13.155&0.017&12.888&0.017&12.746&0.016&---&---&---&12.660&0.034&---&---&---&-1.730&0.190 \\
274&13:26:43.73&-47:22:48.2&RRc&0.311&13.828&0.011&13.758&0.023&13.650&0.022&---&---&---&---&---&---&---&---&---&--- \\
276&13:27:16.51&-47:33:17.6&RRc&0.308&13.727&0.021&13.614&0.046&13.533&0.024&---&---&---&---&---&---&---&---&---&--- \\
280&13:27:09.33&-47:23:05.7&RRc&0.282&13.951&0.012&13.905&0.026&13.816&0.029&---&---&---&---&---&---&---&---&---&--- \\
285&13:25:40.20&-47:34:48.4&RRc&0.329&13.687&0.017&13.504&0.027&13.503&0.015&---&---&---&13.358&0.074&---&---&---&---&--- \\
288&13:28:10.32&-47:23:47.8&RRc&0.295&13.809&0.011&13.719&0.016&13.635&0.019&---&---&---&---&---&---&---&---&---&--- \\
289&13:28:03.68&-47:21:27.9&RRc&0.308&13.743&0.013&13.618&0.015&13.584&0.022&---&---&---&---&---&---&---&---&---&--- \\
357&13:26:17.77&-47:30:23.4&RRc&0.298&13.692&0.027&13.468&0.064&13.468&0.045&13.462&0.044&0.120&13.375&0.041&0.204&---&---&-1.640&0.990 \\
\end{longtable}}
\end{center}
\end{landscape}
\clearpage



\begin{comment}
\section{Extra Figures}
\label{sec:extra_figures}
\begin{figure}
\begin{center}
\includegraphics[width=80mm]{reworked_fitting_code/final_plots/deltadelta_3p6_4p5_spect.pdf}
\caption{$\Delta$3.6~$\mu$m vs. $\Delta$4.5~$\mu$m using spectroscopic metallicities}
\label{fig:deltadelta_spect}
\end{center}
\end{figure}

\begin{figure}
\begin{center}
\includegraphics[width=80mm]{reworked_fitting_code/final_plots/deltadelta_3p6_4p5_phot.pdf}
\caption{$\Delta$3.6~$\mu$m vs. $\Delta$4.5~$\mu$m using photometric metallicities}
\label{fig:deltadelta_phot}
\end{center}
\end{figure}
\end{comment}


%%%%%%%%%%%%%%%%%%%%%%%%%%%%%%%%%%%%%%%%%%%%%%%%%%


% Don't change these lines
\bsp	% typesetting comment
\label{lastpage}
\end{document}

% End of mnras_template.tex
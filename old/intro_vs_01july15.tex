\section{Introduction}
\label{sec:intro}

The Carnegie RR Lyrae Program (CRRP) is a Warm {\it Spitzer} program (PI W. Freedman) with the aim of calibrating the mid--infrared (mid--IR) RR Lyrae period--luminosity (PL) relation. Similar to the Carnegie Hubble Program (CHP) \citep{2011AJ....142..192F}, which used mid--IR observations of Cepheids to measure the Hubble constant \citep[$H_{0}$][]{2012ApJ...758...24F}, the results of the CRRP will be used in to provide an independent, population II calibration of the extragalactic distance scale, hence an independent measurement of $H_{0}$. 

In recent years it has become increasingly important to obtain independent direct measurements of \ho. We are now in the era of precision cosmology, where the measurements of \ho have such small uncertainties that different techniques are no longer all in agreement. For example, the results of  \citet{2011ApJ...730..119R} and \citet{2012ApJ...758...24F}, both which use Cepheids and type Ia supernovae (SNe) as their base, agree very well at $74.4\pm 2.5$~km~s$^{-1}$~Mpc$^{-1}$ and $74.3\pm2.6$~km~s$^{-1}$~Mpc$^{-1}$. However, when we consider the latest results from {\it Planck}, who find $67.48\pm0.98$~km~s$^{-1}$~Mpc$^{-1}$ \citep{2015arXiv150201589P}, there is tension. The {\it Planck} study derives their measurement from a model of the cosmic microwave background (CMB), so is completely independent of the \citeauthor{2011ApJ...730..119R} and \citeauthor{2012ApJ...758...24F} results. 

There have been several recent works that have investigated possible sources of uncertainty in the distance ladder that may contribute to the discrepancy between \ho measurements. For example, \citet{2015ApJ...802...20R} examine the differences in star formation rates in type Ia SNe host galaxies. They find that the intrinsic brightness of a SNe Ia may be affected by the local host environment; i.e. whether the SN occurs in a locally star forming or locally passive environment. \citet{2014MNRAS.440.1138E} reanalysed the Cepheid data from \citet{2011ApJ...730..119R}, and found that different outlier rejection criterea lowered the resultant value of \ho to $70.6 \pm 3.3$~km~s$^{-1}~Mpc${-1}, making it compatible with the value from {\it Planck}. 

The CRRP assess a systematic that was unreachable in the original CHP --- the intrinsic accuracy of the mid--IR Cepheid standard candle distance scale when compared to the standard ruler distance scale of CMB and Baryon Acoustic Oscillation (BAO) measurements. With only one ``test candle'' it is impossible to make any assessment of this accuracy. However, when we have two standard candles with similar precision we can make meaningful comparisons and assess their systematic accuracy.

In this work we focus on the effect of metallicity on the RR Lyrae PL relation. Several Galactic Globular Clusters are being observed as part of CRRP, but $\omega$~Cen is unique in that is exhibits a measureable spread in metallicity \citep{1975ApJ...201L..71F, 2007ApJ...663..296V, 2014ApJ...791..107V}.

There are very few metallic or molecular transition lines in the mid--IR at typical RR Lyrae temperatures, so the effects of metallicity on luminosity should be minimised. However, $\omega$~Cen provides the ideal test bed for any effect that we may not have predicted. Such an effect is not out of the realm of possibility; for example, the CO band head at 4.5~$\mu$m has been found to have a significant dependence on metallicity, and has such prevented the IRAC 4.5~$\mu$m Cepheid observations from being used for distance measurements in the CHP. \citep{2011ApJ...743...76S, 2012ApJ...759..146M, 2015arXiv150206995S}. As our concern in this program is systematic precision, we must ensure that similar effects do not plague the RRL distance scale.  

%% Edited to here

In the past RR Lyrae variables have often been thought of as the poor substitute for Cepheids in terms of distance scale measurements. They are intrinsically fainter, and in the optical follow a much shallower, even horizontal, PL relation. Determining an accurate distance to an RR Lyrae (RRL) in the $V$ band requires knowledge of its metallicity. However, in more recent years near-- and mid--IR observations have shown the true power of RRL as precision distance indicators. In a similar vein to Cepheids, \text{HST} parallaxes were obtained for several Galactic RRL calibrators \citet{2011AJ....142..187B} and several groups have been studying the populations of RRL in globular clusters and nearby dwarf spheroidal galaxies (NEED REFS). 


In the mid--infrared RRL exhibit similar properties to Cepheids \citep{2013ApJ...776..135M}. Their light curve amplitudes are minimised as we are seeing deeper into the star. At the wavelengths observed by Warm \textit{Spitzer} (3.6 and 4.5~$\mu$m) we do not see photospheric effects, but only the effects of temperature driving the pulsation. Essentially, the mid--infrared light curve is tracing the radius change of the star. A by--product of this effect is that the intrinsic width of the RRL PL relation is also minimised in the mid--infrared (mid--IR). The PL relation for pulsational variables can be thought of as a two--dimensional projection of the three--dimensional period--luminosity--colour relation (see figure 3 of \citet{1991PASP..103..933M} for a graphical representation). As the colour--width decreases in the mid--IR, the width of the PL naturally decreases. As one moves from the optical to the mid--IR, the slope of the PL relation steepens and its dispersion dramatically decreases; this phenomenon has been demonstrated in simulations by \citet{2004ApJS..154..633C}, and by several observational efforts, as illustrated in fig. 4 of \citet{2013ApJ...776..135M}. The slope should asymptotically approach the predicted slope of the period--radius relation, resulting in a slope between $-2.4$ and $-2.8$. Through this decrease in dispersion we have found that the intrinsic width of the mid--IR PL for RRL is in fact smaller than for Cepheids -- 0.05~mag compared to 0.10~mag (Monson et al. 2015 [{\bf I couldn't find this reference on the arxiv}], \citep{2015arXiv150507858N}). This translates to an uncertainty on an individual RR Lyrae star of 2\%, compared to 4\% for Cepheids. 



In this work we present the mid--IR PL relation for the RRL in the $\omega$~Cen Galactic Globular Cluster (GGC). 
Here we present a mid--infrared of the RR Lyrae period--luminosity (PL) relation in the IRAC channels 1 and 2 centred on 3.6 and 4.5 \micron\ respectively, as well as a preliminary investigation into metallicity effects on the PL relation.



$\omega$ Cen in particular is ideal for calibrating the RR Lyrae period--luminosity--metallicity relation, as it contains 192 known RR Lyrae \citep{2004A&A...424.1101K} with a metallicity range spanning over 1.5 dex (Bono 2013, priv. comm.); a metallicity spread this wide is not found in any other GGC. As noted in \citet{2006MNRAS.372.1675S}, one of the advantages of using globular clusters to calibrate PL coefficients is that all stars in a cluster can be considered to be at the same distance from Earth. We can therefore assume that any dispersion in the PL relation is a combination of the a) the intrinsic dispersion of the PL relation, b) the photometric uncertainties, and c) dispersion induced by the spread in metallicity of the RRL. We have measured the intrinsic dispersion of the RRL PL from other clusters (e.g. M4, \citet{2015arXiv150507858N}), and our photometric uncertainties are a well defined \textbf{constraint, value?? what is the correct word?}, so the only unknown in this problem is the dispersion due to the spread in metallicity of the cluster. We are lucky with $\omega$~Cen that we can also take a second approach to establishing the metallicity effect on the RRL PL relation. As it is such a unique object, $\omega$~Cen is extremely well studied and many of its RRL have spectroscopic or photometric metallicities available. As another test of the effect of metallicity, we use these measurements to assess the $\gamma$ parameter for the GGC, where 
\begin{equation} \label{eqn:gamma}
\gamma = \dfrac {\Delta \text{mag}} {[Fe/H]}\text{,}
\end{equation}

similar to $\gamma$ used to quantify the effect of metallicity on the zero--point of the Cepheid PL relation. 

The paper is set out as follows: Section~\ref{sec:observations} details the observations and data reduction. Section~\ref{sec:pl_relation} describes the mid--IR PL relations and Section~\ref{sec:distance_moduli} discusses the application of these to a distance measurement of  $\omega$~Cen. Section~\ref{sec:metallicity} and Section~\ref{sec:discussion} examine the effect of metallicity on mid--IR observations of RR Lyrae variables and its implications for distance measurements and the extragalactic distance scale. In Section~\ref{sec:conclusions} we present our conclusions.


%% This paragraph to be moved to the PL relations section %%
The absolute zero--point of the CRRP distance ladder is currently tied to HST fine guidance sensor (FGS) parallax measurements of 5 Galactic RR Lyrae measured by \citet{2011AJ....142..187B}. The {\it Spitzer} observations of these stars, plus an additional 45 Galactic RR Lyrae that will also have their parallaxes measured by {\it Gaia} will be presented in \citet{Monson2015} {\bf REF TO MONSON2015 IN PREP}. 



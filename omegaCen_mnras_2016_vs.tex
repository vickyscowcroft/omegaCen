% mnras_template.tex
%
% LaTeX template for creating an MNRAS paper
%
% v3.0 released 14 May 2015
% (version numbers match those of mnras.cls)
%
% Copyright (C) Royal Astronomical Society 2015
% Authors:
% Keith T. Smith (Royal Astronomical Society)

% Change log
%
% v3.0 May 2015
%    Renamed to match the new package name
%    Version number matches mnras.cls
%    A few minor tweaks to wording
% v1.0 September 2013
%    Beta testing only - never publicly released
%    First version: a simple (ish) template for creating an MNRAS paper

%%%%%%%%%%%%%%%%%%%%%%%%%%%%%%%%%%%%%%%%%%%%%%%%%%
% Basic setup. Most papers should leave these options alone.
\documentclass[a4paper,fleqn,usenatbib]{mnras}

% MNRAS is set in Times font. If you don't have this installed (most LaTeX
% installations will be fine) or prefer the old Computer Modern fonts, comment
% out the following line
% \usepackage{newtxtext,newtxmath}
% Depending on your LaTeX fonts installation, you might get better results with one of these:
%\usepackage{mathptmx}
%\usepackage{txfonts}

% Use vector fonts, so it zooms properly in on-screen viewing software
% Don't change these lines unless you know what you are doing
\usepackage[T1]{fontenc}
\usepackage{ae,aecompl}

%%%%% AUTHORS - PLACE YOUR OWN PACKAGES HERE %%%%%

% Only include extra packages if you really need them. Common packages are:
\usepackage{graphicx}	% Including figure files
\usepackage{amsmath}	% Advanced maths commands
\usepackage{amssymb}	% Extra maths symbols
\usepackage{longtable}
\usepackage{pdflscape}
\usepackage{xspace}
\usepackage{verbatim}

% marconi section 7.1, 7.5
% tailo section 6.1

%%%%%%%%%%%%%%%%%%%%%%%%%%%%%%%%%%%%%%%%%%%%%%%%%%

%%%%% AUTHORS - PLACE YOUR OWN COMMANDS HERE %%%%%

% Please keep new commands to a minimum, and use \newcommand not \def to avoid
% overwriting existing commands. Example:
\newcommand{\ho}{$H_{0}$\xspace}
%\newcommand{\vscomment}[1]{{\bf\textcolor{magenta}{#1}}
\providecommand{\vscomment}[1]{{\textcolor{magenta}{{VS: #1}}}\xspace}
%%%%%%%%%%%%%%%%%%%%%%%%%%%%%%%%%%%%%%%%%%%%%%%%%%

%%%%%%%%%%%%%%%%%%% TITLE PAGE %%%%%%%%%%%%%%%%%%%

% Title of the paper, and the short title which is used in the headers.
% Keep the title short and informative.
\title[Mid-IR RRL PLZ Relations in $\omega$ Cen]{The Carnegie RR Lyrae Program: The Mid-Infrared RR Lyrae Period-Luminosity-Metallicity Relations in $\omega$~Cen}

% The list of authors, and the short list which is used in the headers.
% If you need two or more lines of authors, add an extra line using \newauthor
\author[M.~J.~Durbin et al.]{Meredith~J.~Durbin$^{1,2}$\thanks{E-mail: mdurbin@stsci.edu}
Victoria Scowcroft$^{3}$
Wendy L. Freedman$^{4}$
Barry F. Madore$^{3}$
\newauthor Rachael L. Beaton$^{3}$
%Gurtina Besla$^{5}$ 
%Giuseppe Bono$^{6, 7}$
%Vittorio Braga$^{6, 7}$
%\newauthor Maria-Rosa Cioni$^{8, 9, 10}$
%Gisella Clementini$^{11}$
%Kathryn Johnston$^{12}$
%Nitya Kallivayalil$^{13}$
%\newauthor Juna Kollmeier$^{3}$
%David Law$^{1}$
%Steve Majewski$^{13}$
%Roeland van der Marel$^{1}$
%\newauthor Massimo Marengo$^{14}$
Andrew~J.~Monson$^{5}$
%Jill Neeley$^{14}$
%David Nidever$^{16}$ 
%\newauthor Grzegorz Pietrzynski$^{17, 18}$
%George Preston$^{3}$
Mark Seibert$^{3}$
%Horace Smith$^{19}$
%\newauthor Igor Soszynski$^{17}$
%Andrzej Udalski$^{17}$
\\
% List of institutions
$^1$ Space Telescope Science Institute, 3700 San Martin Drive, Baltimore, MD 21218, USA \\
$^2$ Pomona College, Claremont, CA 91711, USA \\
$^3$ Observatories of the Carnegie Institution of Washington, 813 Santa Barbara St., Pasadena, CA 91101, USA \\
$^4$ Department of Astronomy and Astrophysics, University of Chicago, 5640 S Ellis Ave, Chicago, IL 60637, USA \\
%$^5$ Department of Astronomy and Steward Observatory, University of Arizona, 933 North Cherry Avenue,   Tucson, AZ 85721, USA \\
%$^6$ Univ. Roma ``Tor Vergata", Via della Ricerca Scientifica, 1 - 00133, Roma, Italy \\
%$^7$ INAF-OAR, via Frascati 33 - 00040, Monte Porzio Catone (RM), Italy \\
%$^8$ Universtat Potsdam, Institut fur Physik und Astronomie, Karl-Liebknecht-Str. 24/25, 14476 Potsdam, Germany \\
%$^9$ Leibniz-Institut fur Astrophysik Potsdam, An der Sternwarte 16, 14482 Potsdam, Germany \\
%$^{10}$ University of Hertfordshire, Physics, Astronomy and Mathematics, College Lane, Hatfield AL10 9AB, United Kingdom \\
%$^{11}$ INAF - Osservatorio Astronomico, Via Ranzani n. 1, 40127 Bologna, Italy \\
%$^{12}$ Department of Astronomy, Columbia University, New York, NY 10027, USA  \\
%$^{13}$ Department of Astronomy, University of Virginia, Charlottesville, VA 22904-0818, USA \\
%$^{14}$ Department of Physics and Astronomy, Iowa State University, Ames, IA, USA \\
$^{5}$ Department of Astronomy and Astrophysics, The Pennsylvania State University, 403 Davey Lab, University Park, PA, 16802, USA \\
%$^{16}$ Department of Astronomy, University of Michigan, Ann Arbor, MI 48109, USA \\
%$^{17}$ Warsaw University Observatory Al. Ujazdowskie 4, 00-478 Warszawa, Poland \\
%$^{18}$ Departamento de Astronomia, Universidad de Concepcion, Casilla 160-C, Chile \\
%$^{19}$ Department of Physics and Astronomy, Michigan State University, East Lansing, MI, USA 48824 \\
}

% These dates will be filled out by the publisher
\date{Accepted XXX. Received YYY; in original form ZZZ}

% Enter the current year, for the copyright statements etc.
\pubyear{2016}

% Don't change these lines
\begin{document}
\label{firstpage}
\pagerange{\pageref{firstpage}-\pageref{lastpage}}
\maketitle

% Abstract of the paper
\begin{abstract}
[abstract]
%We present new period-luminosity relations for RR Lyrae variables in 3.6 and 4.5 \micron\ derived from time-resolved IRAC data of $\omega$~Cen. The sample consists of 36 RR Lyrae in 3.6 \micron\ and 37 in 4.5 \micron, 22 of which appear in both channels and have literature values for metallicities. We find no compelling evidence for a metallicity correlation in the residuals, based on a spread of 1.2 dex in [Fe/H].
\end{abstract}

% Select between one and six entries from the list of approved keywords.
% Don't make up new ones.
\begin{keywords}
keyword1 - keyword2 - keyword3
\end{keywords}

%%%%%%%%%%%%%%%%%%%%%%%%%%%%%%%%%%%%%%%%%%%%%%%%%%

%%%%%%%%%%%%%%%%% BODY OF PAPER %%%%%%%%%%%%%%%%%%

%% IMPORTANT NOTES FROM VS:

% MNRAS is a UK journal - change your spellcheck language in your editor to British English. 
% Correct plural of RR Lyrae is RR Lyrae variables (technically, singular should be RR Lyrae variable, as RR Lyrae itself is a named object)
% Use 1 dash - for a minus sign in math mode
% Use 2 dashes -- to hyphenate words
% Use 3 dashes --- to put a dash between parts of a sentence or to denote a minus sign outside of math mode

% ^^^ From the MNRAS style guide:

% Hyphens (one dash in LaTeX) should be used for compound adjectives (e.g. low-density gas, least-squares fit, two-component model). This also applies to simple adjectival units (e.g. 1.5-m telescope, 284.5-nm line), but not to complex units or ranges, which could become cumbersome (e.g. 15 km sÐ1 feature, 100Ð200 µm observations). Some words (e.g. time-scale) are always hyphenated as part of journal style (see below). 

% N-rules (two dashes in LaTeX): these are longer than hyphens and are used (i) to separate key words, (ii) as parentheses (e.g. the results Ð assuming no temperature gradient Ð are indicative of É), (iii) to denote a range (e.g. 1.6Ð2.2 µm), and (iv) to denote the joining of two words (e.g. KolmogorovÐSmirnov test, HerbigÐHaro object). 

% M-rules (three dashes in TeX/LaTeX) are not used in MNRAS.

% Figures and tables can go at their appropriate places in the document rather than at the end. 
% To update bibtex source run ads_importer.py omegaCen_mnras_2015 after latex, then run latex, bibtex, latex, latex (ads_importer.py is available from VS's github, rely's on ADS style refs).

\section{Introduction}
\label{sec:intro}

%Suggested intro structure from RLB:
%
%\begin{enumerate}
%	\item $H_{0}$ and the need for other distance ladders. $\checkmark$
%	\item RR Lyrae as a good local standard candle, particularly in the mid-IR $\checkmark$
%	\item MidIR RRL as a standard candle that needs auxilliray analysis in order to be applied
%	\item The need to constrain metallicity effects, using the well known horizontal branch dependencies on metallicity as a basis
%	\item One such analysis $\rightarrow$ empirical study of metallicity term in the NIR/MIR - literature not settled.
%	\item Murareva et al. (2015), table 3 for NIR
%	\item This work (Durbin et al. 2016) explores the MIR metallicity effect {\bf which has not previously been explored empirically}
%	\item Establish $\omega$ Cen as the best place to do this test
%	\item Be clear that prior analyses of $\omega$ Cen have been inconclusive in optical compared to theoretical expectations
%	\item Outline of paper.
%\end{enumerate}
%
%{\bf Remeber to go from BIG PICTURE to FINE DETAIL. Go from the WHY to the HOW.}

The Carnegie RR Lyrae Program (CRRP) is a Warm {\it Spitzer} program \citep[][PID 90002]{2012sptz.prop90002F} which aims to provide an independent, population II measurement of the Hubble constant ($H_{0}$), tied to RR Lyrae variables with high precision geometric distances in the Milky Way. Similar to the Carnegie Hubble Program \citep[CHP][]{2011AJ....142..192F}, CRRP will provide a single instrument measurement of $H_{0}$. This will provide important constraints on the external accuracy of the standard candle distance ladder, and the internal consistency of the distance measurements of Cepheids and RR Lyrae variables.

In the era of `precision cosmology' it is important to fully understand all sources of uncertainty in our experiments. Although the results distance ladder measurements such as \citet{2011ApJ...730..119R} and \citet{2012ApJ...758...24F} agree very well, 
%at $74.4\pm 2.5$~km~s$^{-1}$~Mpc$^{-1}$ and $74.3\pm2.6$~km~s$^{-1}$~Mpc$^{-1}$, respectively, 
when we consider the latest results from {\it Planck} 
%who find $67.48\pm0.98$~km~s$^{-1}$~Mpc$^{-1}$ \citep{2015arXiv150201589P},
 there is tension. The {\it Planck} study derives its measurement from a model of the cosmic microwave background (CMB), so is completely independent of the \citeauthor{2011ApJ...730..119R} and \citeauthor{2012ApJ...758...24F} results. Works such as \citet{2015ApJ...802...20R} and \citet{2014MNRAS.440.1138E} have examined the contribution of systematic uncertainties at the far end of the distance ladder to this tension. The CRRP is quantifying the systematic uncertainty in the standard candle distance ladder by making meaningful comparisons at the base of the Cepheid and RR Lyrae distance ladders in the mid--infrared, where distance measurements of similar precision are achievable. 

A good local standard candle \citep[as defined by][]{1986ApJ...303....1A} has the following  features: a)~a physical basis, b)~objective measurables, c)~minimal corrections, and d)~small scatter. RR Lyrae variables (hereafter RRL) are excellent standard candles in the mid--infrared. As \citet{1986MNRAS.220..279L} demonstrated, unlike at optical wavelengths where their absolute magnitudes depend only on metallicity, in the infrared RRL follow a clear period--luminosity (PL) relation. The PL relation has a physical basis (i.e. the PL relation is directly linked to the period--radius relation), and both the period and luminosity of the RRL can be objectively measured. In the mid--IR minimal extinction corrections are required \citep[$A_{[3.6]} \approx A_{V}/16$,][]{2005ApJ...619..931I}, and the dispersion of the relation in the infrared has been shown to be small compared to optical wavelengths \citep{2015ApJ...808...50M, 2004ApJS..154..633C}. Through this decrease in dispersion we have found that the intrinsic width of the mid-IR PL for RRL is in fact \emph{smaller} than for Cepheids -- less than 0.05~mag compared to 0.10~mag \citep{2015ApJ...808...11N, 2011ApJ...743...76S}. This translates to an uncertainty on an individual RRL of below 2\% for RRL, compared to 4\% for Cepheids at 3.6~$\mu$m. Thus, for nearby systems RRL are the most precise standard candles.

Although observing RRL at mid--IR wavelengths dramatically reduces the effects of reddening and extinction, the consequences of this shift to longer wavelengths on other parameters in the empirical RRL PL relation are yet to be determined. The most important factor that must be considered is the effect of metallicity. Both theory and observation have demonstrated that the position of the horizontal branch in the optical colour--magnitude diagram is dependent on metallicity \citep[e.g.][]{2015ApJ...808...50M, 2004ApJS..154..633C, 2003MNRAS.344.1097B, 1994AJ....108..222N}. However, the size of the metallicity contribution in the in the infrared PL relation is yet to be settled in the literature, either from a theoretical or empirical standpoint. 

The most in--depth study of the effect of metallicity on the mid--IR PL relation to date comes from \citet{2015PKAS...30..183D}, who used the ALLWISE data release \citep{2010AJ....140.1868W, 2014yCat.2328....0C} to examine possible changes in the RRL PL relation in globular clusters with different metallicities. They found a moderate dependence of [W1] (approximately equivalent to {\it Spitzer} [3.6]) with [Fe/H] of $\gamma_{W1} = 0.102$~mag~dex$^{-1}$, slightly larger than their value for near--IR, and approximately half that of the optical dependence ($\gamma_{K} = 0.088$~mag~dex$^{-1}$, $\gamma_{V} = 0.232$~mag~dex$^{-1}$). The observational work of \citet{2015ApJ...807..127M} appears to confirm this result, favouring a low value of $\gamma_K$ for Milky Way and LMC RRL. However, the empirical study of \citet{2015AJ....150...90K} considers near--IR observations of RRL in the Carina dwarf Spheroidal (dSph) galaxy, this time testing different PL relations with a range of metallicity coefficient. Their results favour a larger value for $\gamma_K$ than \citet{2015PKAS...30..183D}, with average values of $\gamma_K \approx 0.18$~mag~dex$^{-1}$. This is consistent with the theoretical models by \citet{2015ApJ...808...50M}, who find that the metallicity dependence of the infrared PL relation is just as large as at optical wavelengths.

In this work we focus on the effects of metallicity on the RRL PL relation in the mid-IR. Several Galactic Globular Clusters are being observed as part of CRRP, but $\omega$~Cen is unique in that it exhibits a measurable spread in metallicity, with the most recent results estimating $0.8~\leq~\Delta~\text{[Fe/H]}~\leq~1.4$~dex \citep{2014ApJ...791..107V, 2012ApJ...746...14M, 2010ApJ...722.1373J}. This makes $\omega$~Cen the ideal site for a metallicity study, as all the RRL can be considered to be at the same distance, leaving the metallicities of the individual stars as the only free parameter. $\omega$~Cen has been used previously for such studies, ranging from empirical tests in the optical \citep[e.g.][]{2003MNRAS.345...86O, 1991ApJ...373L..43L} and near--IR \citep[e.g.][]{2006MmSAI..77..245C, 2006ApJ...652..362D}, to semi--empirical tests using population synthesis techniques \citep{2016MNRAS.tmp..104T}. Our study is unique as it is the first study to use the RRL population of $\omega$~Cen to empirically measure the metallicity effect on the mid--IR RRL PL relation.


%RR Lyrae variables (hereafter RRL) are intrinsically fainter than Cepheids, and in the optical follow a much shallower, even horizontal, PL relation \citep{2004ApJS..154..633C}. Determining an accurate distance to an RRL in the $V$ band requires knowledge of its metallicity. However, \citet{1986MNRAS.220..279L} showed that the true power of RRL as distance indicators lies in the IR passbands. Recently, everal groups have been studying the populations of RRL in globular clusters and nearby dwarf spheroidal galaxies in the infrared \citep[e.g.][and references therein]{2013ApJ...767...62G, 2014ApJ...786..147O, 2015ApJ...806..200C, 2015A&A...578A.128K}. Several Galactic RRL now have high precision geometric distances measured as part of the HST-Fine Guidance Sensor (HST-FGS) parallax program \citep{2011AJ....142..187B}, and can therefore be used to calibrate the absolute zero-point of the RRL PL relation.% and \textit{HST} parallaxes were obtained for several Galactic RRL calibrators \citep{2011AJ....142..187B}, where calibrators are RRL whose distances can be determined by both parallax and period-luminosity fitting. {\bf (Is that right? Should I be more general and say any object whose distance can be determined by more than one distance ladder method?)}

%Distance measurements made in the mid-IR benefit from reduced extinction effects, where $A_{[3.6]}$ and $A_{[4.5]}$ are 16 to 20 times lower than $A_V$ \citep{1989ApJ...345..245C, 2005ApJ...619..931I}. Additionally, the precision of distances obtained from the RRL PL relation is increased. At the wavelengths observed by Warm \textit{Spitzer} (3.6 and 4.5~$\mu$m) we do not see photospheric effects, but only the effects of temperature driving the pulsation; essentially, the mid-infrared light curve is tracing the change in radius of the star through the pulsation cycle. A by-product of this effect is that the intrinsic width of the RRL PL relation is also minimised in the mid-IR. The PL relation for pulsational variables can be thought of as a two-dimensional projection of the three-dimensional period-luminosity-colour relation (see figure 3 of \citet{1991PASP..103..933M} for a graphical representation). As the effects of temperature are reduced, the colour-width of the PL decreases in the mid-IR. As one moves from the optical to the mid-IR, the slope of the PL relation steepens and its dispersion dramatically decreases; the slope should asymptotically approach the predicted slope of the period-radius relation, resulting in a slope between $-2.4$ and $-2.8$, confirmed empirically by \citet{2013ApJ...776..135M}. Through this decrease in dispersion we have found that the intrinsic width of the mid-IR PL for RRL is in fact \emph{smaller} than for Cepheids -- less than 0.05~mag compared to 0.10~mag \citep{2015ApJ...808...11N}. This translates to an uncertainty on an individual RRL of below 2\%, compared to 4\% for Cepheids. Thus, for nearby systems RRL are the most precise standard candles.
%
%In this work we focus on the effects of metallicity on the RRL PL relation in the mid-infrared. Several Galactic Globular Clusters are being observed as part of CRRP, but $\omega$~Cen is unique in that it exhibits a measurable spread in metallicity \citep{1975ApJ...201L..71F, 2007ApJ...663..296V, 2014ApJ...791..107V}.
%
%There are very few metallic or molecular transition lines in the mid-IR at typical RRL temperatures, so the effects of metallicity on luminosity should be minimised. However, $\omega$~Cen provides the ideal environment for a direct empirical test. Such an effect is not out of the realm of possibility; for example, the strength of the CO band head at 4.5~$\mu$m has been found to have a significant effect on Cepheid colours \citep{2010ApJ...709..120M}, and has such prevented the IRAC 4.5~$\mu$m Cepheid observations from being used for distance measurements in the CHP \citep{2011ApJ...743...76S, 2012ApJ...759..146M, 2016ApJ...816...49S}. Our concern in this program is systematic precision, and therefore we must test for any potential effects directly.
%
The paper is set out as follows: Section~\ref{sec:observations} details the observations and data reduction. Section~\ref{sec:results} presents the photometry of the $\omega$~Cen RRL. Section~\ref{sec:pl_relation} describes the mid-IR PL relations and Section~\ref{sec:distance_moduli} discusses the application of these to a distance measurement of  $\omega$~Cen. Section~\ref{sec:metallicity} examines the effects of metallicity on RRL magnitudes and distance estimates. Section~\ref{sec:discussion} discusses the implications of this, and other systematic effects we consider in this work. In Section~\ref{sec:conclusions} we present our conclusions.

\section{Observations \& Data Reduction}
\label{sec:observations}
This work combines mid-IR observations from the Warm {\it Spitzer} mission \citep[][PID 90002]{2012sptz.prop90002F} with supporting near-IR observations from the FourStar instrument on the Baade-Magellan telescope at Las Campanas Observatory \citep{2013PASP..125..654P}. Figure~\ref{fig:omegaCen_fields} shows a $K_s$ FourStar image with the {\it Spitzer} fields outlined, the positions of known RRab plotted as circles, and the positions of known RRc as triangles.

\begin{figure*}
\begin{center}
\includegraphics[width=160mm, trim=1cm 0 1cm 0]{reworked_fitting_code/final_plots/omegacen_coverage_map_new.pdf}
\caption{A $K_s$-band image of $\omega$~Cen from the FourStar camera, overlaid with a catalog of RRL from \citet{2004A&A...424.1101K} and footprints of the {\it Spitzer} IRAC fields. The circular points are RRab's and the triangular points are RRc's; we adopt this convention throughout the paper. The three black rectangle outlines are the IRAC field of view for each pointing in the 3.6~$\mu$m channel, and the white rectangle outlines show the same for 4.5~$\mu$m.} 
\label{fig:omegaCen_fields}
\end{center}
\end{figure*}

\subsection{Warm {\em Spitzer} Data}
\label{sec:spitzer_reduction}
The Warm~\textit{Spitzer} observations for this work were taken as part of the CRRP. Three fields in $\omega$~Cen were chosen; their positions and the positions of known $\omega$~Cen RRLs are shown in Figure~\ref{fig:omegaCen_fields}. To obtain optimal RRL light curves we observed each field 12 times over approximately 16 hours, roughly corresponding to the period of the longest period RRL we expected in the field. The observations of all three fields were taken on 2013 May 10 and 2013 May 11. Each field was observed using the {\it Spitzer} InfraRed Array Camera (IRAC) \citep{2004ApJS..154...10F} with a 30~s frame time with a medium scale, and gaussian 5-point dither pattern to mitigate any image artefacts. Images were collected in both the 3.6 and 4.5~$\mu$m channels. 
%% VS added info about field shapes and missing colour information.
The elongated field shapes come from the design of IRAC; while the [3.6] channel is collecting on-target data, the [4.5] channel collects off target data ``for free'', and vice versa. We chose to include these off-target fields to maximise the number of RRL in our final sample and to increase the legacy value of our data set to the community. 

The science images were created using \textsc{mopex} \citep{2006SPIE.6274E..0CM}, first running overlap correction on the corrected basic calibrated data frames (cBCDs) then mosaicking them at 0.6 arcsec pixel scale using the drizzle algorithm. Mosaicked location-correction images were created at the same time. 

%% VS updated the photometry procedure text
PSF photometry was performed using {\sc daophot} and {\sc allframe} \citep{1987PASP...99..191S, 1994PASP..106..250S}. The PSF model was created for each field/filter combination using the first epoch data and was applied to every epoch. As the observations were taken temporally close together the effects of telescope rotation between epochs on the mosaicked PSF were minimal, so making a single good PSF model for each field/filter combination was much more efficient than creating one for every epoch. 

Master star lists for {\sc allframe} were created for each filter/field combination using a median mosaicked image created by {\sc mopex}. We did not use the same single master star list for both filters as only a small proportion ($1/3$) of the [3.6] and [4.5] fields overlap each other. Instead we performed separate {\sc allframe} reductions for each filter, and combined the results after the fact using {\sc daomatch} and {\sc daomaster}. Our mid-IR photometry is calibrated to the standard system set by \citet{2005PASP..117..978R}.

\subsection{FourStar Data}
\label{sec:fourstar_reduction}

$J$, $H$ and $K_s$ data were taken with the FourStar instrument on the Baade-Magellan telescope at Las Campanas Observatory \citep{2013PASP..125..654P} on the nights of 2013 June 25, 2013 June 27, and 2013 June 28. Four epochs were obtained each night in each filter for a total of 12 epochs. A mosaic of $5\times3$ slightly overlapping pointings (tiles), each with FourStar's native $10.9 \times 10.9$ arcminute field of view, covered a $50\times30$ arcminute field of view centred on $\omega$~Cen. Each tile consists of a 5 point dither pattern with a 5.8 second exposure time. Stacked mosaics of the entire field were made as well as individual tiles using a customised pipeline for FourStar data. The purpose of the individual tiles is to provide photometry with better time resolution than the large mosaic. 

PSF photometry of the tiles was performed using \textsc{daophot} and \textsc{allframe} \citep{1987PASP...99..191S, 1994PASP..106..250S}. A PSF model was created for each epoch/tile/filter combination. A master star list for \textsc{allframe} was created from the final $K_s$ mosaic and the multi-wavelength/epoch results were combined using \textsc{daomatch} and \textsc{daomaster}. Our final photometry is calibrated to the 2MASS standard system \citep{2006AJ....131.1163S}. 

\subsection{Crowding}
\label{sec:crowding}

The primary limiting factor in the photometric precision is crowding. To assess crowding for individual RRLs, we compared the {\it Spitzer} images to the FourStar $K_s$-band image. The 0.159 arcsec/pixel resolution of the $K_s$ band image compared to the 0.6 arcsec/pixel resolution of the IRAC images enabled us to more accurately determine which stars were significantly contaminated. RRL were determined to be contaminated if there were one or more resolved stars in the $K$-band image within a 3.6 arcsecond (6 IRAC pixels) radius of the RRL. 77 RRLs out of the original catalog of 192 \citep{2004A&A...424.1101K} were rejected due to crowding, another 13 were outside the FourStar mosaic field of view, and four more were found to be unusable due to contamination.
%Our full, uncrowded RRL sample consists of 97 stars in $J$ and $H$, 99 in $K_s$, 37 in 3.6~$\mu$m, and 43 in 4.5~$\mu$m. {\bf [CHECK THIS]}

\section{Results}
\label{sec:results}

Our final photometry catalog, including magnitudes and uncertainties for $J\!H\!K_s$, [3.6], and [4.5], is presented in Table~\ref{tab:everything}. {\bf Note: this will not be an appendix in the final draft, we're just putting it there for now because it's huge.}
The average magnitudes presented in Table~\ref{tab:everything} are flux averages, and the photometric uncertainties of the time series data are the error on the mean. %\vscomment{I commented out the bit about gloess fitting - it didn't seem to fit.}
%\textsc{gloess} fitting \citet{2004AJ....128.2239P} was also attempted on the light curves as done in previous CHP works \citep[e.g.][]{2011ApJ...743...76S, 2012ApJ...759..146M, 2014ApJ...794..107R, 2016ApJ...816...49S}, but the phase coverage is sufficiently extensive that it did not significantly improve results.

Our full, uncrowded RRL sample consists of 96 stars in $J$ and $H$, 98 in $K_s$, 38 in [3.6], and 42 in [4.5]; the small number of stars in the IRAC bands compared to the FourStar bands is due to the smaller coverage of the IRAC pointings (see Figure~\ref{fig:omegaCen_fields}). For the PL fitting, detailed Section~\ref{sec:pl_relation}, we use only the stars for which we have photometry in all five bandpasses, ensuring that the same range of periods and metallicities are sampled for each wavelength. This helps to reduce any biases that may be introduced by non--uniform sampling in the distance moduli fits in Section~\ref{sec:distance_moduli}. Our final RRL sample consists of 24 stars, with 12 in each pulsation mode.

\section{Period-Luminosity Relations}
\label{sec:pl_relation}

We fit PL relations using the theoretical near-infrared PL relation parameters presented in \citet{2015ApJ...808...50M} for the $JHK_s$ bands, and the empirical PL relation parameters derived from photometry of RRLs in the globular cluster M4 (NGC 6121) from \citet{2015ApJ...808...11N} for the IRAC bands. With the use of preexisting PL relation coefficients, the distance modulus becomes the only free parameter in our fit. We fit all distance moduli using an unweighted least-squares method, and fit the distance modulus to each pulsation mode in each wavelength separately. We also refine the fit by sigma-clipping the residuals of the [3.6] fit at a $2\sigma$ level, resulting in the rejection of two more stars.

\begin{figure*}
\begin{center}
\includegraphics[width=160mm]{reworked_fitting_code/final_plots/multiwavelength_PL_m4_clipped.pdf}
\caption{PL relations for $J\!H\!K_s$, [3.6], and [4.5] photometry assuming [Fe/H]$=-1.677$. Here circles represent RRab stars and triangles represent RRc's. Coloured points are the final consistent sample with photometry in all 5 wavebands and grey points are stars that did not appear in all bands. The unfilled points are stars rejected from the final sample based on $2\sigma$ clipping of the PL residuals in [3.6].}
\label{fig:omegaCen_pl_m4}
\end{center}
\end{figure*}

For the mid-IR we use the PL relations from \citet{2015ApJ...808...11N}, as described in Table~\ref{tab:pl_table_m4}. These relations take the form
\begin{equation}M = a + b \times (\log (P) + P_0) \end{equation}
where $a$ and $b$ are empirically derived coefficients and $P_0$ is the absolute value of the logarithm of the mean period of the M4 RRL sample. We calculate the absolute PL zero-points adopting \citeauthor{2015ApJ...808...11N}'s M4 distance modulus of $\mu=11.399$ mag.

%% PL equations

\begin{table}
\centering
\caption{Empirical mid-IR RRL period-luminosity relation coefficients \citep{2015ApJ...808...11N}, for relations of the form $M = a + b \times (\log (P) + P_0)$ with intrinsic dispersion $\sigma$. These relations are derived from RRL in the globular cluster M4.} 
\label{tab:pl_table_m4}
\begin{tabular}{l||c|c|c|c|c|r} 
\hline \hline
Band & Mode & $a$ & $b$ & $P_0$ & $\sigma$ \\
\hline
$[3.6]$ & RRab & $-0.558$ & $-2.370$ & $0.260$ & $0.040$ \\
            & RRc & $-0.192$ & $-2.658$ & $0.550$ & $0.079$ \\
$[4.5]$ & RRab & $-0.593$ & $-2.355$ & $0.260$ & $0.045$ \\
            & RRc & $-0.240$ & $-2.979$ & $0.550$ & $0.057$ \\ 
            \hline
\end{tabular}
\end{table}

The $JHK_s$ RRL PL relations are described in Table~\ref{tab:pl_table_theo}. The relations take the form
\begin{equation}M = a + b\times\log P + c\times[\text{Fe/H}]\end{equation}
where $a$, $b$, and $c$ are theoretically derived coefficients.

\begin{table}
\centering
\caption{Theoretical near-IR RRL period-luminosity relation coefficients \citep{2015ApJ...808...50M}, for relations of the form $M = a + b \times \log P + c \times [\text{Fe/H}]$ with intrinsic dispersion $\sigma$.}
\label{tab:pl_table_theo}
\begin{tabular}{l||c|c|c|c|c|r} 
\hline \hline
Band & Mode & $a$   & $b$   & $c$   & $\sigma$ \\
\hline
$J$ & RRab & $-0.510$ & $-1.980$ & $0.170$ & $0.060$ \\
       & RRc & $-1.070$ & $-2.460$ & $0.150$ & $0.040$ \\
$H$ & RRab & $-0.760$ & $-2.240$ & $0.190$ & $0.040$\\
       & RRc & $-1.310$ & $-2.700$ & $0.160$ & $0.020$\\
$K_s$ & RRab & $-0.820$ & $-2.270$ & $0.180$ & $0.030$\\
           & RRc & $-1.370$ & $-2.720$ & $0.150$ & $0.020$ \\       
\hline
\end{tabular}
\end{table}

The theoretical PL relations for the near-IR have a metallicity-dependent term; however, we do not have known metallicities for all RRL in our sample. We therefore use the average [Fe/H] of the RRLs for which there are known metallicities. Using spectroscopic metallicities from \citet{2006ApJ...640L..43S}, we obtain an average [Fe/H] of $-1.677$. \vscomment{Here is where you should introduce Figure~\ref{fig:metallicity_hists} to demonstrate the applicability of your chosen metallicity.}

\section{Distance Moduli}
\label{sec:distance_moduli}

We combine the uncorrected distance moduli from each bandpass to obtain a mean reddening value and reddening-corrected distance modulus. We fit the near-infrared reddening law from \citet{1989ApJ...345..245C} to the $JHK_s$ data and the mid-infrared law from \citet{2005ApJ...619..931I} to [3.6] and [4.5] simultaneously, assuming a ratio of total to selective absorption $R_V = 3.1$. The resulting fit is shown in Figure~\ref{fig:omegaCen_dist_m4_mean}. We derive a true mean dereddened distance modulus of $\langle \mu_0 \rangle = 13.789 \pm 0.018$ with $E(B-V) = 0.084 \pm 0.030$ using the weighted mean RRab + RRc distances. The individual uncorrected distance moduli $\mu$, corrected distance moduli $\mu_0$, and PL residuals are shown in Table~\ref{tab:dist_mod}.

\begin{figure}
\begin{center}
\includegraphics[width=80mm, trim=0.75cm 0 0.5cm 0]{reworked_fitting_code/final_plots/multiwavelength_distance_m4_clipped_mean.pdf}
\caption{Top: Uncorrected distance moduli for the final sample of $J\!H\!K_s$, [3.6], and [4.5] photometry. Filled circles are the mean distance moduli using both RRab and RRc stars, open circles are the distance moduli using only RRab stars, and open triangles are distance moduli using only RRc stars. Here the NIR and MIR reddening laws are fit to the mean distance moduli. The solid and dashed horizontal lines are the mean corrected distance modulus and its $1\sigma$ errors respectively. Bottom: reddening-corrected distance moduli and mean corrected distance modulus. Errors on the corrected distance moduli are the quadrature sum of the uncorrected distance moduli errors and the reddening error at the requisite wavelength.}
\label{fig:omegaCen_dist_m4_mean}
\end{center}
\end{figure}

\begin{figure}
\begin{center}
\includegraphics[width=80mm, trim=0.75cm 0 0.5cm 0]{reworked_fitting_code/final_plots/multiwavelength_distance_m4_clipped_ab.pdf}
\caption{Same as Figure~\ref{fig:omegaCen_dist_m4_mean}, with the reddening laws fit to the RRab distances only instead of the mean.}
\label{fig:omegaCen_dist_m4_ab}
\end{center}
\end{figure}

\begingroup
\setlength{\tabcolsep}{.5em}

\begin{table*}
\centering
\caption{Uncorrected distance moduli $\mu$, corrected distance moduli $\mu_0$, and the PL dispersion $\sigma$. The corrected distance moduli $\mu_0$ are equal to $\mu - A_\lambda A_V$, where $A_V$ is derived from fitting the reddening laws to the mean distance moduli.}
\label{tab:dist_mod}
\begin{tabular}{l||c|c|c|c|c|c|c|c|r} 
\hline \hline
Band & $\mu$, RRab & $\mu$, RRc & $\mu$, mean & $\mu_0$, RRab & $\mu_0$, RRc & $\mu_0$, mean & $\sigma_{\text{PL}}$, RRab & $\sigma_{\text{PL}}$, RRc \\
\hline
$J$ & $13.865 \pm 0.023$ & $13.866 \pm 0.019$ & $13.865 \pm 0.015$ & $13.788 \pm 0.036$ & $13.790 \pm 0.033$ & $13.789 \pm 0.031$ & 0.127 & 0.082 \\
$H$ & $13.821 \pm 0.032$ & $13.844 \pm 0.026$ & $13.832 \pm 0.021$ & $13.773 \pm 0.037$ & $13.796 \pm 0.031$ & $13.784 \pm 0.027$ & 0.170 & 0.107 \\
$K_s$ & $13.816 \pm 0.025$ & $13.831 \pm 0.021$ & $13.824 \pm 0.016$ & $13.786 \pm 0.027$ & $13.801 \pm 0.024$ & $13.794 \pm 0.020$ & 0.147 & 0.089 \\
$[3.6]$ & $13.750 \pm 0.026$ & $13.838 \pm 0.031$ & $13.794 \pm 0.020$ & $13.733 \pm 0.027$ & $13.821 \pm 0.031$ & $13.777 \pm 0.021$ & 0.124 & 0.096 \\
$[4.5]$ & $13.771 \pm 0.052$ & $13.890 \pm 0.061$ & $13.830 \pm 0.040$ & $13.756 \pm 0.052$ & $13.875 \pm 0.061$ & $13.816 \pm 0.040$ & 0.154 & 0.219 \\
\hline
\end{tabular}
\end{table*}

\begin{comment}
$J$ & $13.865 \pm 0.023$ & $13.866 \pm 0.019$ & $13.865 \pm 0.015$ & $13.788 \pm 0.036$ & $13.790 \pm 0.033$ & $13.789 \pm 0.031$ & 0.127 & 0.082 \\
$H$ & $13.821 \pm 0.032$ & $13.844 \pm 0.026$ & $13.832 \pm 0.021$ & $13.773 \pm 0.037$ & $13.796 \pm 0.031$ & $13.784 \pm 0.027$ & 0.170 & 0.107 \\
$K_s$ & $13.816 \pm 0.025$ & $13.831 \pm 0.021$ & $13.824 \pm 0.016$ & $13.786 \pm 0.027$ & $13.801 \pm 0.024$ & $13.794 \pm 0.020$ & 0.147 & 0.089 \\
$[3.6]$ & $13.750 \pm 0.026$ & $13.838 \pm 0.031$ & $13.794 \pm 0.020$ & $13.733 \pm 0.027$ & $13.821 \pm 0.031$ & $13.777 \pm 0.021$ & 0.124 & 0.096 \\
$[4.5]$ & $13.771 \pm 0.052$ & $13.890 \pm 0.061$ & $13.830 \pm 0.040$ & $13.756 \pm 0.052$ & $13.875 \pm 0.061$ & $13.816 \pm 0.040$ & 0.154 & 0.219 \\

$[3.6]$ & RRab & 0.124 & 0.040 & 0.052 & 0.106 \\
$[3.6]$ & RRc & 0.096 & 0.079 & 0.044 & 0.034 \\
$[4.5]$ & RRab & 0.154 & 0.045 & 0.046 & 0.140 \\
$[4.5]$ & RRc & 0.219 & 0.057 & 0.054 & 0.205 \\
\end{comment}

\endgroup

\begin{table}
\centering
\caption{Corrected distance moduli $\mu_0$ using the $A_V$ value derived from fitting the reddening laws to only the RRab distance moduli.}
\label{tab:dist_mod_ab}
\begin{tabular}{l||c|c|c|r} 
\hline \hline
Band & $\mu_0$, RRab & $\mu_0$, RRc & $\mu_0$, mean \\
\hline
$J$ & $13.739 \pm 0.047$ & $13.741 \pm 0.045$ & $13.740 \pm 0.043$ \\
$H$ & $13.742 \pm 0.041$ & $13.766 \pm 0.036$ & $13.754 \pm 0.033$ \\
$K_s$ & $13.767 \pm 0.030$ & $13.782 \pm 0.026$ & $13.775 \pm 0.023$ \\
$[3.6]$ & $13.722 \pm 0.028$ & $13.810 \pm 0.032$ & $13.766 \pm 0.022$ \\
$[4.5]$ & $13.747 \pm 0.053$ & $13.866 \pm 0.061$ & $13.806 \pm 0.041$ \\
\hline
\end{tabular}
\end{table}

\begin{comment}
& $13.739 \pm 0.047$ & $13.741 \pm 0.045$ & $13.740 \pm 0.043$ \\
& $13.742 \pm 0.041$ & $13.766 \pm 0.036$ & $13.754 \pm 0.033$ \\
& $13.767 \pm 0.030$ & $13.782 \pm 0.026$ & $13.775 \pm 0.023$ \\
& $13.722 \pm 0.028$ & $13.810 \pm 0.032$ & $13.766 \pm 0.022$ \\
& $13.747 \pm 0.053$ & $13.866 \pm 0.061$ & $13.806 \pm 0.041$ \\
\end{comment}

It is apparent from Figure~\ref{fig:omegaCen_dist_m4_mean} that there are large discrepancies in the distance moduli in [3.6] and [4.5] for the two pulsation modes; these discrepancies contribute to the relatively low $E(B-V)$ value and high dereddened distance modulus compared to previously determined values \citep[e.g.][]{2002ASPC..265...95L, 2006ApJ...652..362D}. 
If we remove the RRc's and fit the extinction curve only to the RRab's, as shown in Figure~\ref{fig:omegaCen_dist_m4_ab}, we obtain a better fit of all points to the extinction curve than when we use the mean. From these distance moduli we derive a true dereddened distance modulus of $\langle \mu_0 \rangle = 13.743 \pm 0.026$ with $E(B-V) = 0.138 \pm 0.044$, both of which are closer to accepted values than the values derived from the weighted mean distance moduli. All individual corrected distance moduli from this fit are shown in Table~\ref{tab:dist_mod_ab}.

Given the large errors in the [4.5] distances, we also fit the extinction curve to the $JHK_s$ and [3.6] distances only, excluding [4.5] entirely; this was found to have a negligible effect on the final distance modulus and reddening for both the mean and RRab-only measurements.

\section{Metallicity}
\label{sec:metallicity}

\vscomment{This is where I think some more reading would help. There's been several more papers in the last few months that I hadn't had time to keep up with until the last week or so. See the separate list I'm sending you. Summarising, I think that the intro paragraph to this section can be extended to discuss the uncertainty around the size of the [Fe/H] effect. I think that is where the most recent literature is pointing.}

Theoretical models suggest that the metallicity dependence of the RRL PL relation should decrease monotonically from the optical to the near-infrared \citep{2001MNRAS.326.1183B, 2004ApJS..154..633C}. Observational evidence corroborates this; previous investigations performed on WISE data suggest no obvious metallicity dependence in the mid-IR PL relations \citep{2013ApJ...776..135M}.

$\omega$ Cen is ideal for examining the RRL period-luminosity-metallicity relation, because it is known to have a large spread in metallicity \citep[$0.8~\leq~\Delta~\text{[Fe/H]}~\leq~1.4$~dex][]{2014ApJ...791..107V, 2012ApJ...746...14M, 2010ApJ...722.1373J}.
A metallicity spread this wide is not found in any other Galactic globular cluster. One of the advantages of using globular clusters to calibrate PL coefficients is that all stars in a cluster can be considered to be at the same distance from Earth. The dispersion in the PL relation is a combination of the a) the intrinsic dispersion of the PL relation, b) the photometric uncertainties, and c) dispersion induced by other complicating factors such as the spread in metallicity. Since we have measured the intrinsic dispersion of the RRL PL in [3.6] and [4.5] from the cluster M4 \citep{2015ApJ...808...11N} and our photometric uncertainties are well understood, we can isolate the remaining scatter due to astrophysical sources such as metallicity. %is the potential dispersion due to the spread in metallicity of the cluster.

We know from mid-IR spectra that a significant CO feature sits within the IRAC [4.5] filter. In the case of Cepheids, \citet{arxiv:1603.03776} have shown that this has a significant effect on the [4.5] magnitudes, and is metallicity dependent. However, this effect decreases with increasing temperatures, turning off completely above 6000~K where all the CO has been destroyed \citep{arxiv:1603.03776}. As even the coolest RRL have temperatures over 6000~K \citep{1971PASP...83..697I}, we expect to see no such CO absorption in the [4.5] PL relation, nor do we expect any other metallicity effects. However, we can directly and empirically test this prediction. %If there are any other unanticipated metallicity effects for RRL, they must be smaller than the dispersion of the PL relations themselves, but we must still perform empirical tests to search for such effects.

\subsection{Metallicity Contribution to the Overall Dispersion}
\label{sec:dispersions}

We can place an upper limit on the contribution of metallicity to the $\omega$~Cen PL dispersion using the known variances of the observed distribution of $\omega$~Cen PL residuals $\sigma_{\text{observed}}^2$, the intrinsic PL width $\sigma_{\text{intrinsic}}^2$, and the scatter induced by photometric error $\sigma_{\text{phot}}^2$. The scatter contributed by metallicity can therefore be constrained as follows:

\begin{equation}
\sigma_\text{[Fe/H]} \leq \sqrt{\sigma_{\text{observed}}^2 - \sigma_{\text{intrinsic}}^2 - \sigma_{\text{phot}}^2}
\end{equation}

The calculated $\sigma_\text{[Fe/H]}$ values for all IRAC PL relations are shown in Table~\ref{tab:metallicity_sigma}, along with with $\sigma_{\text{observed}}$ and the other scatter components.

\begin{table}
\centering
\caption{The standard deviation of the observed spread of the PL residuals $\sigma_{\text{observed}}$ ($\sigma_{\text{PL}}$ in Table~\ref{tab:dist_mod}) and its components: $\sigma_{\text{intrinsic}}$ ($\sigma$ in Table~\ref{tab:pl_table_m4}), $\sigma_{\text{phot}}$, and $\sigma_{[\text{Fe/H}]}$ for all IRAC PL relations.}
\label{tab:metallicity_sigma}
\begin{tabular}{lccccr} 
\hline \hline
Band & Mode & $\sigma_{\text{observed}}$ & $\sigma_{\text{intrinsic}}$ & $\sigma_{\text{phot}}$ & $\sigma_{[\text{Fe/H}]}$ \\
\hline
$[3.6]$ & RRab & 0.124 & 0.040 & 0.052 & 0.106 \\
$[3.6]$ & RRc & 0.096 & 0.079 & 0.044 & 0.034 \\
$[4.5]$ & RRab & 0.154 & 0.045 & 0.046 & 0.140 \\
$[4.5]$ & RRc & 0.219 & 0.057 & 0.054 & 0.205 \\
\hline
\end{tabular}
\end{table}
\begin{comment}
$[3.6]$ & RRab & 0.088 & 0.040 & 0.048 & 0.062 \\ % & $0.282$ & $0.234$\\
$[3.6]$ & RRc & 0.102 & 0.079 & 0.041 & 0.050 \\ %& $0.778$ & $0.323$\\
$[4.5]$ & RRab & 0.173 & 0.045 & 0.046 & 0.160 \\ % & $0.712$ & $0.590$\\
$[4.5]$ & RRc & 0.202 & 0.057 & 0.048 & 0.188 \\ %& $1.660$ & $0.689$\\
\end{comment}

We use these \textcolor{magenta}{estimates} of $\sigma_\text{[Fe/H]}$ to assess the $\gamma$ parameter for $\omega$~Cen, where 
\begin{equation} \label{eqn:gamma}
\gamma = \dfrac {\Delta \text{mag}} {\Delta [\text{Fe/H}]}\text{ mag dex} ^{-1}\text{,}
\end{equation}
\vscomment{adding units to ur eqns\\}
similar to $\gamma$ used to quantify the effect of metallicity on the zero-point of the Cepheid PL relation \citep{1998ApJ...498..181K, 2009MNRAS.396.1287S}. Here we calculate $\gamma$ in units of mag dex$^{-1}$ by dividing the standard deviation of the metallicity component of the PL scatter by the standard deviation of the metallicity distribution for both \citep{2006ApJ...640L..43S} and photometric \citep{2000AJ....119.1824R} metallicities; the results for each PL are shown in Table~\ref{tab:gamma}. We use all available metallicity values for each pulsation mode when taking the standard deviation, as we do not have metallicity values for all the stars in our samples, although the metallicity values we do have trace the overall metallicity distributions fairly well (see Figure~\ref{fig:metallicity_hists}).

For [3.6] we find $\gamma \leq 0.247$ mag dex$^{-1}$ for RRab's, and $\gamma \leq 0.383$ mag dex$^{-1}$ for RRc's.

\begin{table}
\centering
\caption{Metallicity standard deviations and $\gamma$ values.}
\label{tab:gamma}
\begin{tabular}{lccccr} 
\hline \hline
Band & Mode & $\sigma_{\text{spect}}$ & $\sigma_{\text{phot}}$ & $\gamma_{\text{spect}}$ & $\gamma_{\text{phot}}$ \\
\hline
$[3.6]$ & RRab & 0.250 & 0.254 & 0.247 & 0.243 \\ % & $0.282$ & $0.234$\\
$[3.6]$ & RRc & 0.132 & 0.243 & 0.383 & 0.208 \\ %& $0.778$ & $0.323$\\
$[4.5]$ & RRab & 0.250 & 0.254 & 0.642 & 0.632 \\ % & $0.712$ & $0.590$\\
$[4.5]$ & RRc & 0.132 & 0.243 & 1.421 & 0.772 \\ %& $1.660$ & $0.689$\\
\hline
\end{tabular}
\end{table}

\vscomment{We talked about this on Friday so I'm not going to beat a dead horse, just a reminder to check these numbers. I went through and checked the $\gamma$ values and what I worked out came out big. They're upper limits on $\gamma$, and just because they're big doesn't mean we need to worry. It's interesting. It's also why I want you to read a couple of the papers and see what you think of the values you get when you consider what else could be going on with the horizontal branch.\\ I've just had a thought. Can you also do this for J, H, K? It would be interesting to see what $\gamma$ came out for those using this method. I think it's still a valid technique because we're using a single metallicity for the whole cluster rather than the individual metallicities for each star in the PL fitting. Humour me. }

\subsection{PL Residuals and Individual Metallicities}
\label{sec:residuals}

$\omega$~Cen provides a second approach to testing for a metallicity effect on the RRL PL relation. The cluster is extremely well studied and many of its RRL have spectroscopic or photometric metallicities in the literature \citep[e.g.][]{2006ApJ...640L..43S, 2000AJ....119.1824R}. 

%If there is any correlation between [Fe/H] and the PL residuals, we expect it to be a linear one, consistent with the theoretical metallicity terms in the $JHK_s$ PL relations, $c\times[\text{Fe/H}]$.
Theory predicts a linear metallicity term in the PLZ relation for the near-IR, $c\times[\text{Fe/H}]$.
We thus fit a relation of the form
\begin{equation}
\Delta\text{mag} = \gamma \times[\text{Fe/H}] + d
\end{equation}
to the [3.6] and [4.5] PL residuals and metallicity values for stars with known individual metallicity values, as shown in Figure~\ref{fig:metallicity_residuals}. The scatter in the [3.6] and [4.5] PL relations is higher for $\omega$~Cen than it is for M4 \citep{2015ApJ...808...11N, 2015ApJ...799..165B}; however, when we examine [Fe/H] vs. the residuals of each PL relation, $\gamma$ is within $1\sigma$ of zero for all fits, indicating that there is no significant metallicity dependence in the PL residuals, consistent with predictions.

% this is further demonstrated in figures 7 and 8 where we plot the PL relations with RRL color coded by metallicity. DESCRIBE THE THING. as you can see, there is no trend of metallicity with direction of residual, further demonstrating that metallicity is not driving factor of scatter
% more stars in H band with [fe/h], PL relation more studied, minimal metallicity effects also in that band
% further weight to conclusion that metallicity is not the thing

More accurate metallicities will be required to constrain any potential effect further.


\begin{figure*}
\begin{center}
\includegraphics[width=160mm]{reworked_fitting_code/final_plots/metallicity_hists.pdf}
\caption{Histograms of spectroscopic \citep[right column]{2006ApJ...640L..43S} and photometric \citep[left column]{2000AJ....119.1824R} [Fe/H] values for RRab (top row) and RRc (bottom row). The light blue histograms represent all known metallicity values for each type, and the dark blue are the metallicity values in our final RRL sample. The number $N$ at the top right corner of each subplot is the number of stars in our final sample that have known metallicity values for the given type and metallicity catalog.}
\label{fig:metallicity_hists}
\end{center}
\end{figure*}


\begin{figure*}
\begin{center}
\includegraphics[width=160mm]{reworked_fitting_code/final_plots/metallicity_vs_residuals_m4_split.pdf}
\caption{Photometric \citep{2000AJ....119.1824R} and spectroscopic \citep{2006ApJ...640L..43S} [Fe/H] values vs. period-luminosity residuals for RRab and RRc in [3.6] and [4.5]. Solid lines are the line of best fit with slope $\gamma$, and dashed lines are the $1\sigma$ confidence intervals. The $\gamma$ parameter from equation 4 in the top right corner of each subplot. All $\gamma$ values are consistent with zero.}
\label{fig:metallicity_residuals}
\end{center}
\end{figure*}


%\begin{comment}

\begin{figure*}
\begin{center}
\includegraphics[width=160mm]{reworked_fitting_code/final_plots/spect_color_PL.pdf}
\caption{The $H$-band PL relation with fundamentalised RRc periods, with colour indicating spectroscopic \citep[top]{2006ApJ...640L..43S} and photometric \citep[bottom]{2000AJ....119.1824R} metallicity values. Grey points have no known metallicity values.}
\label{fig:hband}
\end{center}
\end{figure*}

\begin{figure*}
\begin{center}
\includegraphics[width=160mm]{reworked_fitting_code/final_plots/phot_color_PL.pdf}
\caption{The $H$-band PL relation with fundamentalised RRc periods, with colour indicating spectroscopic \citep[top]{2006ApJ...640L..43S} and photometric \citep[bottom]{2000AJ....119.1824R} metallicity values. Grey points have no known metallicity values.}
\label{fig:hband}
\end{center}
\end{figure*}

%\end{comment}

\section{Discussion}
\label{sec:discussion}

% why is [4.5] different than [3.6]

% include period-color plot

% move feh histograms up to after PL plot

\begin{comment}
\begin{figure}
\begin{center}
\includegraphics[width=80mm]{reworked_fitting_code/final_plots/metallicity_comparison_all_clipped.pdf}
\caption{Spectroscopic vs. photometric measurements of [Fe/H] for RRLs in $\omega$~Cen.}
\label{fig:metallicity_comparison}
\end{center}
\end{figure}
\end{comment}

% 1. Add discussion of why gamma in 6.1 is big but gamma in 6.2 is small Ñ age/mass spread rather than metalliicity effect. Can refer to new plot here too.

% 2. Why the discrepancy between RRab and RRc - data effect rather than intrinsic??


While we find high upper bounds on the contribution of metallicity to the PL scatter in the IRAC passbands, particularly [4.5], it is unlikely that this scatter is due to metallicity alone. \citet{1986A&A...169..111G} and \citet{1991ApJ...373L..43L} demonstrate that RRL luminosity is dependent on the horizontal branch morphology as well as metallicity; it is also well known that $\omega$~Cen contains RRLs in multiple evolutionary states \citep{2008MmSAI..79..342S, 2015A&A...577A..99N}.

Results from the {\em GAIA} mission \citep{1996A&AS..116..579L} are expected to improve the overall characterisation of the RRL PL relation dramatically. Trigonometric parallaxes and spectrophotometric metallicities of Galactic RRLs from {\em GAIA} will increase the number of calibrators for the absolute RRL PL relations by an order of magnitude \citep[{\bf overview paper}]{2012MNRAS.426.2463L}. {\bf something about omega cen parallaxes and metallicities too}

We also anticipate that the NIRCam instrument on {\em JWST} \citep{2005SPIE.5904...21B, 2006SSRv..123..485G} will provide substantial improvements over IRAC for investigations of this nature. The NIRCam filters F356W and F444W will provide data in passbands comparable to IRAC's [3.6] and [4.5] at an order of magnitude higher resolution (0.065 arcsec/pixel), which will significantly decrease photometric error due to crowding and therefore allow us to obtain data for all known RRLs in $\omega$~Cen. 

% 1. Add discussion of why gamma in 6.1 is big but gamma in 6.2 is small Ñ age/mass spread rather than metallicity effect. Can refer to new plot here too.

% 2. Why the descrepancy between RRab and RRc - data effect rather than intrinsic??


% 2015A&A...577A..99N 

% Eric Persson's metallicity omega cen project - 'we explore a subset of RR Lyrae variables in the w Centauri globular cluster for which a large set of stellar metallicities will be available for a broad range of RR Lyrae' overview paper
% spectroscopically with magellan

% zpt changes by .15 mag depending on whether you include RRc in galactic calibrator sample so ?????

\begin{comment}

Suggestions from what we discussed today:

* How does the choice of metallicity affect the results?
* Comparison of the photometric and spectroscopic metallicities - discussion of Fig 5
Ñ Also note figures must appear in the order they are referred to in the text.
* Could crowding have affected the result? Probably not systematically, BUT it did reduce the sample of RRL dramatically.
* How does the calibration of the PL affect the result?
* Better Calibration of the PL relation
* Future metallicity measurements
** GAIA
* Better resolution in the midÑIR
** JWST
VS will send references for GAIA, JWST separately

\end{comment}

\section{Conclusions}
\label{sec:conclusions}

We derive a true mean dereddened distance modulus for $\omega$~Cen of $\langle \mu_0 \rangle = 13.739 \pm 0.024$ with reddening $E(B-V) = 0.110 \pm 0.042$ using distance moduli derived from RRab PL relations in $JHK_s$, [3.6], and [4.5]. Using the mean of RRab and RRc distance moduli in the same passbands, we derive a true mean dereddened distance modulus of $\langle \mu_0 \rangle = 13.781 \pm 0.018$ with reddening $E(B-V) = 0.066 \pm 0.030$. The RRab results are closer to literature values.

We also constrain the contribution of metallicity, quantified as $\gamma = \Delta \text{mag} / \Delta [\text{Fe/H}]$, to the scatter of the $\omega$~Cen PL relations by subtracting the variances of the intrinsic scatter and photometric error from the observed PL variance and taking the square root of the result, and dividing that by the standard deviation of the distribution of metallicity values. We also measure $\gamma$ directly using the slope of the linear fit to the PL residuals vs. [Fe/H].

% the name of the trashcan is astronomy
% zizek.gif

\section*{Acknowledgements}
\label{sec:acknowledgements}

We thank Eric Persson for his many contributions to this project.

This work is based on observations made with the Spitzer Space Telescope, which is operated by the Jet Propulsion Laboratory, California Institute of Technology under a contract with NASA. Support for this work was provided by NASA through an award issued by JPL/Caltech.

This work was also supported in part by the Claremont-Carnegie Astrophysics Research Program.

This publication makes use of data products from the Two Micron All Sky Survey, which is a joint project of the University of Massachusetts and the Infrared Processing and Analysis Center/California Institute of Technology, funded by the National Aeronautics and Space Administration and the National Science Foundation.

This research has made use of the NASA/IPAC Extragalactic Database (NED), which is operated by the Jet Propulsion Laboratory, California Institute of Technology, under contract with the National Aeronautics and Space Administration.


%%%%%%%%%%%%%%%%%%%%%%%%%%%%%%%%%%%%%%%%%%%%%%%%%%

%%%%%%%%%%%%%%%%%%%% REFERENCES %%%%%%%%%%%%%%%%%%

% The best way to enter references is to use BibTeX:

\bibliographystyle{mnras}
\bibliography{omegaCen_mnras_2016}
 % if your bibtex file is called example.bib


% Alternatively you could enter them by hand, like this:
% This method is tedious and prone to error if you have lots of references

%%%%%%%%%%%%%%%%%%%%%%%%%%%%%%%%%%%%%%%%%%%%%%%%%%

%%%%%%%%%%%%%%%%% APPENDICES %%%%%%%%%%%%%%%%%%%%%

\clearpage
%\begin{comment}

\newpage

\appendix

%% To get the appendix table numbering right you need to have a section after the appendix command. This way the table will be referred to as Table A1, rather than Table 1 (which you already have another one of in the main text).

% Add comments to table
\setlength{\tabcolsep}{.5em}

\section{RRL Photometry}
\label{sec:phot_table_appendix}
\onecolumn
\begin{landscape}
\begin{center}
\scriptsize{
\begin{longtable}{lcccccccccccccccccccr}
\caption{Parameters for 99 RRLs in $\omega$~Cen. First five columns are star ID, right ascension and declination, pulsation mode, and period in days from \citet{2004A&A...424.1101K}. Columns 6 through 11 are $J\!H\!K_s$ apparent magnitudes and errors from FourStar data. Columns 12 through 17 are 3.6~$\mu$m and 4.5~$\mu$m apparent magnitudes, errors, and PL residuals ($\Delta [3.6]$ and $\Delta [4.5]$) from IRAC data. Columns 18-21 are photometric ([Fe/H], p) and spectroscopic ([Fe/H], s) metallicities and errors from \citet{2000AJ....119.1824R} and \citet{2006ApJ...640L..43S} respectively.
\label{tab:everything}} 
\tabularnewline 
ID & RA (J2000) & Dec (J2000) & Mode & $P$ (days) & $J$  & $\sigma_{J}$ & $H$  & $\sigma_{H}$ & $K_s$  & $\sigma_{K_s}$ & [3.6] & $\sigma_{{[3.6]}}$ & $\Delta [3.6]$ & [4.5] & $\sigma_{{[4.5]}}$ & $\Delta [4.5]$ & [Fe/H], p   & $\sigma_{[\text{Fe/H}]}$, p & [Fe/H], s & $\sigma_{[\text{Fe/H}]}$, s \\
\hline
\endfirsthead
\multicolumn{4}{c}%
{\tablename\ \thetable\ - \textit{Continued from previous page}} \\
\hline 
ID & RA (J2000) & Dec (J2000) & Mode & $P$ (days) & $J$  & $\sigma_{J}$ & $H$  & $\sigma_{H}$ & $K_s$  & $\sigma_{K_s}$ & [3.6] & $\sigma_{{[3.6]}}$ & $\Delta [3.6]$ & [4.5] & $\sigma_{{[4.5]}}$ & $\Delta [4.5]$ & [Fe/H], p   & $\sigma_{[\text{Fe/H}]}$, p & [Fe/H], s & $\sigma_{[\text{Fe/H}]}$, s \\
\hline
\endhead
\hline \multicolumn{4}{r}{\textit{Continued on next page}} \\
\endfoot
\hline
\endlastfoot
%id&ra&dec&type&per&mag_j&merr_j&mag_h&merr_h&mag_k&merr_k&mag_3&merr_3&resid_m4_3&mag_4&merr_4&resid_m4_4&photfeh&photfeh_err&spectfeh&spectfeh_err \\
3&13:25:56.15&-47:25:53.8&RRab&0.841&13.247&0.017&12.982&0.018&12.882&0.017&12.841&0.039&-0.087&12.708&0.036&0.035&-1.540&0.050&---&--- \\
4&13:26:12.93&-47:24:18.8&RRab&0.627&13.475&0.016&13.219&0.021&13.133&0.020&13.030&0.036&0.026&13.026&0.035&0.016&-1.740&0.050&---&--- \\
5&13:26:18.33&-47:23:12.4&RRab&0.515&13.700&0.017&13.549&0.020&13.507&0.027&13.387&0.043&-0.129&13.340&0.030&-0.096&-1.350&0.080&-1.240&0.110 \\
7&13:27:00.90&-47:14:00.5&RRab&0.713&13.333&0.009&13.151&0.031&13.036&0.018&---&---&---&---&---&---&-1.460&0.080&---&--- \\
8&13:27:48.45&-47:28:20.3&RRab&0.521&13.505&0.015&13.258&0.017&13.223&0.014&---&---&---&---&---&---&-1.910&0.280&---&--- \\
9&13:25:59.58&-47:26:24.0&RRab&0.523&13.776&0.017&13.534&0.021&13.470&0.016&13.315&0.036&-0.072&13.279&0.039&-0.051&-1.490&0.060&---&--- \\
10&13:26:06.99&-47:24:36.6&RRc&0.375&13.579&0.014&13.395&0.023&13.345&0.019&13.342&0.037&-0.026&13.168&0.037&0.112&-1.660&0.100&---&--- \\
11&13:26:30.59&-47:23:01.6&RRab&0.565&13.481&0.014&13.307&0.028&13.219&0.025&13.050&0.058&---&---&---&---&-1.670&0.130&-1.610&0.220 \\
12&13:26:27.21&-47:24:06.2&RRc&0.387&13.590&0.018&13.379&0.028&13.305&0.025&13.168&0.048&0.112&13.448&0.088&-0.208&-1.530&0.140&---&--- \\
13&13:25:58.18&-47:25:21.6&RRab&0.669&13.353&0.019&13.081&0.022&13.058&0.017&12.918&0.032&0.072&12.860&0.031&0.117&-1.910&0.000&---&--- \\
14&13:25:59.74&-47:39:09.6&RRc&0.377&13.588&0.011&13.343&0.020&13.365&0.016&---&---&---&13.299&0.045&---&-1.710&0.130&---&--- \\
15&13:26:27.11&-47:24:38.0&RRab&0.811&13.245&0.018&13.020&0.031&12.954&0.025&13.149&0.084&---&---&---&---&-1.640&0.390&-1.680&0.180 \\
16&13:27:37.69&-47:37:34.8&RRc&0.330&13.680&0.015&13.502&0.022&13.437&0.018&---&---&---&---&---&---&-1.290&0.080&-1.650&0.460 \\
18&13:27:45.11&-47:24:56.6&RRab&0.622&13.371&0.010&13.131&0.024&13.100&0.016&13.006&0.043&---&---&---&---&-1.780&0.280&---&--- \\
20&13:27:14.05&-47:28:06.3&RRab&0.616&13.410&0.015&13.210&0.036&13.125&0.025&13.060&0.039&0.016&12.940&0.029&0.122&---&---&-1.520&0.340 \\
21&13:26:11.17&-47:25:58.8&RRc&0.381&13.578&0.016&13.399&0.027&13.361&0.020&13.301&0.047&-0.003&13.200&0.032&0.061&-0.900&0.110&---&--- \\
22&13:27:41.04&-47:34:07.6&RRc&0.396&13.572&0.012&13.380&0.016&13.288&0.017&---&---&---&---&---&---&-1.630&0.170&-1.600&0.990 \\
23&13:26:46.50&-47:24:39.5&RRab&0.511&13.941&0.025&13.794&0.048&13.658&0.033&13.325&0.064&---&---&---&---&-1.080&0.140&-1.350&0.580 \\
24&13:27:38.32&-47:34:14.5&RRc&0.462&13.419&0.012&13.218&0.014&13.138&0.014&---&---&---&---&---&---&-1.860&0.030&---&--- \\
30&13:26:15.94&-47:29:56.0&RRc&0.404&13.521&0.021&13.287&0.046&13.251&0.030&13.188&0.047&0.041&13.071&0.060&0.112&-1.750&0.170&-1.620&0.280 \\
32&13:27:03.32&-47:21:38.9&RRab&0.620&13.508&0.009&13.244&0.018&13.132&0.018&---&---&---&---&---&---&-1.530&0.160&---&--- \\
33&13:25:51.60&-47:29:05.8&RRab&0.602&13.338&0.015&13.106&0.022&13.091&0.019&---&---&---&13.006&0.035&---&-2.090&0.230&-1.580&0.420 \\
34&13:26:07.21&-47:33:10.4&RRab&0.734&13.273&0.014&13.018&0.014&12.916&0.013&---&---&---&12.838&0.065&---&-1.710&0.000&---&--- \\
35&13:26:53.21&-47:22:34.7&RRc&0.387&13.586&0.012&13.463&0.024&13.356&0.023&---&---&---&---&---&---&-1.560&0.080&-1.630&0.360 \\
36&13:27:10.11&-47:15:29.8&RRc&0.380&13.534&0.007&13.372&0.019&13.307&0.014&---&---&---&---&---&---&-1.490&0.230&---&--- \\
38&13:27:03.30&-47:36:30.2&RRab&0.779&13.226&0.015&12.943&0.019&12.814&0.018&---&---&---&---&---&---&-1.750&0.180&-1.640&0.400 \\
39&13:27:59.77&-47:34:42.3&RRc&0.393&13.560&0.009&13.415&0.014&13.308&0.014&---&---&---&---&---&---&-1.960&0.290&---&--- \\
40&13:26:24.56&-47:30:46.2&RRab&0.634&13.517&0.022&13.250&0.051&13.153&0.033&13.062&0.049&-0.017&13.416&0.056&-0.385&-1.600&0.080&-1.620&0.190 \\
44&13:26:22.39&-47:34:35.3&RRab&0.568&13.677&0.014&13.425&0.023&13.368&0.018&---&---&---&13.132&0.036&---&-1.400&0.120&-1.290&0.350 \\
45&13:25:30.88&-47:27:21.0&RRab&0.589&13.513&0.015&13.201&0.015&13.164&0.014&---&---&---&13.070&0.028&---&-1.780&0.250&---&--- \\
46&13:25:30.23&-47:25:51.8&RRab&0.687&13.299&0.016&12.998&0.017&12.947&0.014&---&---&---&---&---&---&-1.880&0.170&---&--- \\
47&13:25:56.46&-47:24:12.0&RRc&0.485&13.420&0.020&13.223&0.018&13.150&0.018&13.099&0.030&-0.080&13.073&0.026&-0.126&-1.580&0.310&---&--- \\
49&13:26:07.78&-47:37:55.5&RRab&0.605&13.566&0.012&13.238&0.019&13.220&0.016&---&---&---&13.099&0.049&---&-1.980&0.110&---&--- \\
50&13:25:53.94&-47:27:35.8&RRc&0.386&13.647&0.014&13.402&0.015&13.362&0.014&---&---&---&13.305&0.056&---&-1.590&0.190&---&--- \\
51&13:26:42.66&-47:24:21.4&RRab&0.574&13.597&0.014&13.378&0.033&13.270&0.029&13.315&0.083&---&---&---&---&-1.640&0.210&-1.840&0.230 \\
54&13:26:23.54&-47:18:47.7&RRab&0.773&13.281&0.016&12.998&0.017&12.954&0.015&12.799&0.030&---&---&---&---&-1.660&0.120&-1.800&0.230 \\
56&13:25:55.53&-47:37:44.1&RRab&0.568&13.643&0.009&13.386&0.022&13.353&0.017&---&---&---&13.232&0.035&---&-1.260&0.150&---&--- \\
57&13:27:49.38&-47:36:50.5&RRab&0.794&13.234&0.015&12.995&0.018&12.882&0.014&---&---&---&---&---&---&-1.890&0.140&---&--- \\
58&13:26:13.05&-47:24:03.0&RRc&0.370&13.660&0.017&13.495&0.018&13.421&0.021&13.345&0.033&-0.013&13.309&0.034&-0.011&-1.370&0.180&-1.910&0.310 \\
59&13:26:18.43&-47:29:46.7&RRab&0.519&13.727&0.023&13.424&0.043&13.391&0.033&13.248&0.071&0.004&13.418&0.064&-0.181&-1.000&0.280&---&--- \\
63&13:25:07.96&-47:36:54.1&RRab&0.826&13.223&0.017&12.862&0.017&12.869&0.012&---&---&---&---&---&---&-1.730&0.090&---&--- \\
64&13:26:02.22&-47:36:19.2&RRc&0.344&13.638&0.013&13.438&0.022&13.407&0.022&---&---&---&13.314&0.044&---&-1.460&0.230&---&--- \\
66&13:26:33.08&-47:22:25.2&RRc&0.407&13.542&0.011&13.359&0.022&13.264&0.020&13.103&0.035&---&---&---&---&-1.680&0.340&---&--- \\
67&13:26:28.62&-47:18:46.9&RRab&0.564&13.610&0.014&13.384&0.016&13.326&0.015&13.368&0.047&---&---&---&---&-1.100&0.000&-1.190&0.230 \\
68&13:26:12.80&-47:19:35.7&RRc&0.535&13.258&0.021&13.004&0.015&12.970&0.015&12.928&0.050&---&---&---&---&-1.600&0.010&---&--- \\
69&13:25:11.02&-47:37:33.5&RRab&0.635&---&---&---&---&13.112&0.014&---&---&---&---&---&---&-1.520&0.140&---&--- \\
70&13:27:27.76&-47:33:42.7&RRc&0.391&13.529&0.013&13.282&0.029&13.254&0.022&---&---&---&---&---&---&-1.940&0.150&-1.740&0.300 \\
72&13:27:33.11&-47:16:22.9&RRc&0.385&13.554&0.010&13.339&0.017&13.311&0.014&---&---&---&---&---&---&-1.320&0.220&---&--- \\
73&13:25:53.75&-47:16:10.8&RRab&0.575&13.480&0.018&13.251&0.017&13.215&0.016&---&---&---&---&---&---&-1.500&0.090&---&--- \\
74&13:27:07.22&-47:17:33.9&RRab&0.503&13.622&0.008&13.457&0.016&13.405&0.015&---&---&---&---&---&---&-1.830&0.360&---&--- \\
75&13:27:19.70&-47:18:46.5&RRc&0.422&13.410&0.011&13.175&0.028&13.137&0.025&---&---&---&---&---&---&-1.490&0.080&-1.820&0.990 \\
76&13:26:57.23&-47:20:07.7&RRc&0.338&13.634&0.012&13.488&0.017&13.449&0.020&---&---&---&---&---&---&-1.450&0.130&---&--- \\
77&13:27:20.89&-47:22:05.6&RRc&0.426&13.474&0.013&13.264&0.028&13.199&0.021&---&---&---&---&---&---&-1.810&0.000&-1.840&0.430 \\
79&13:28:24.99&-47:29:25.2&RRab&0.608&13.382&0.010&13.162&0.016&13.123&0.015&---&---&---&---&---&---&-1.390&0.180&---&--- \\
81&13:27:36.68&-47:24:48.3&RRc&0.389&13.542&0.013&13.326&0.033&13.286&0.025&13.248&0.076&---&---&---&---&-1.720&0.310&-1.990&0.430 \\
82&13:27:35.61&-47:26:30.3&RRc&0.336&13.579&0.016&13.324&0.024&13.296&0.018&---&---&---&13.827&0.104&---&-1.560&0.200&-1.710&0.560 \\
83&13:27:08.42&-47:21:34.1&RRc&0.357&13.603&0.010&13.431&0.024&13.370&0.022&---&---&---&---&---&---&-1.300&0.220&---&--- \\
84&13:24:47.45&-47:29:56.5&RRab&0.580&---&---&12.833&0.017&12.781&0.016&---&---&---&---&---&---&-1.470&0.100&---&--- \\
85&13:25:06.49&-47:23:34.0&RRab&0.743&13.344&0.011&---&---&---&---&---&---&---&---&---&---&-1.870&0.310&---&--- \\
94&13:25:57.06&-47:22:46.1&RRc&0.254&14.070&0.024&13.934&0.022&13.870&0.027&13.858&0.038&-0.092&13.799&0.029&-0.014&-1.000&0.110&---&--- \\
95&13:25:24.95&-47:28:53.2&RRc&0.405&13.497&0.015&13.269&0.017&13.264&0.017&---&---&---&13.178&0.024&---&-1.840&0.550&---&--- \\
97&13:27:08.49&-47:25:30.9&RRab&0.692&13.302&0.010&13.143&0.029&13.034&0.022&12.964&0.061&-0.008&12.702&0.064&0.240&-1.560&0.370&-1.740&0.170 \\
101&13:27:30.24&-47:29:51.0&RRc&0.341&13.708&0.016&13.484&0.030&13.436&0.023&---&---&---&---&---&---&-1.880&0.320&---&--- \\
102&13:27:22.11&-47:30:12.3&RRab&0.691&13.320&0.012&13.033&0.022&12.993&0.020&12.984&0.049&-0.028&13.056&0.072&-0.113&-1.840&0.130&-1.650&0.160 \\
103&13:27:14.29&-47:28:36.3&RRc&0.329&13.620&0.018&13.409&0.040&13.377&0.034&12.960&0.071&---&13.024&0.066&---&-1.920&0.110&-1.780&0.270 \\
104&13:28:07.76&-47:33:44.9&RRab&0.867&13.732&0.096&13.626&0.154&13.452&0.141&---&---&---&---&---&---&-1.830&0.180&---&--- \\
105&13:27:46.02&-47:32:43.9&RRc&0.335&13.768&0.014&13.615&0.020&13.533&0.018&---&---&---&---&---&---&-1.240&0.180&---&--- \\
107&13:27:14.05&-47:30:57.9&RRab&0.514&13.597&0.017&13.340&0.038&13.301&0.030&13.535&0.219&---&13.351&0.076&---&-1.360&0.110&---&--- \\
115&13:26:12.30&-47:34:17.5&RRab&0.630&13.401&0.012&13.176&0.017&13.103&0.013&---&---&---&---&---&---&-1.870&0.010&-1.640&0.320 \\
117&13:26:19.91&-47:29:21.0&RRc&0.422&13.480&0.020&13.274&0.043&13.202&0.031&13.110&0.044&0.071&12.949&0.043&0.179&-1.680&0.250&---&--- \\
120&13:26:25.52&-47:32:48.6&RRab&0.549&13.525&0.049&13.072&0.079&13.135&0.094&12.958&0.066&0.236&12.927&0.055&0.253&-1.390&0.060&-1.150&0.160 \\
121&13:26:28.17&-47:31:50.5&RRc&0.304&13.741&0.016&13.648&0.033&13.531&0.026&13.414&0.037&0.144&13.302&0.033&0.249&-1.460&0.130&-1.830&0.400 \\
122&13:26:30.31&-47:33:02.2&RRab&0.635&13.369&0.018&13.132&0.042&13.062&0.024&13.057&0.052&-0.013&12.987&0.052&0.043&-2.020&0.180&-1.790&0.210 \\
123&13:26:51.17&-47:37:13.2&RRc&0.474&13.462&0.016&13.239&0.019&13.174&0.017&---&---&---&---&---&---&-1.640&0.010&---&--- \\
124&13:26:54.49&-47:39:07.5&RRc&0.332&13.708&0.013&13.510&0.018&13.482&0.023&---&---&---&---&---&---&-1.330&0.230&---&--- \\
125&13:26:48.92&-47:41:03.7&RRab&0.593&13.420&0.015&13.200&0.016&13.153&0.015&---&---&---&---&---&---&-1.670&0.220&-1.810&0.380 \\
126&13:28:08.03&-47:40:46.7&RRc&0.342&13.642&0.011&13.467&0.017&13.370&0.016&---&---&---&---&---&---&-1.310&0.130&---&--- \\
127&13:25:19.36&-47:28:37.6&RRc&0.305&---&---&---&---&13.579&0.018&---&---&---&13.573&0.063&---&-1.590&0.080&---&--- \\
128&13:26:17.75&-47:30:13.0&RRab&0.835&13.207&0.018&12.927&0.032&12.810&0.020&---&---&---&12.445&0.074&---&-1.880&0.040&---&--- \\
130&13:26:09.93&-47:13:40.0&RRab&0.493&13.688&0.021&13.527&0.032&13.418&0.025&---&---&---&---&---&---&-1.460&0.170&---&--- \\
147&13:27:15.86&-47:31:09.2&RRc&0.423&13.397&0.012&12.934&0.041&13.083&0.022&---&---&---&12.585&0.096&---&-1.660&0.140&---&--- \\
149&13:27:32.94&-47:13:43.6&RRab&0.683&13.354&0.015&13.061&0.035&13.024&0.024&---&---&---&---&---&---&-1.210&0.240&---&--- \\
150&13:27:40.21&-47:36:00.1&RRab&0.899&13.068&0.019&12.757&0.025&12.692&0.018&---&---&---&---&---&---&-1.760&0.340&---&--- \\
151&13:28:25.40&-47:16:00.2&RRab&0.408&13.501&0.013&13.301&0.020&13.265&0.016&---&---&---&---&---&---&-1.300&0.240&---&--- \\
163&13:25:49.42&-47:20:21.5&RRc&0.313&13.763&0.019&13.557&0.016&13.545&0.025&---&---&---&---&---&---&-1.180&0.270&---&--- \\
168&13:25:52.78&-47:32:02.9&RRc&0.321&14.176&0.015&14.000&0.020&13.960&0.018&---&---&---&---&---&---&---&---&---&--- \\
169&13:27:20.47&-47:23:59.1&RRc&0.319&13.805&0.013&13.735&0.019&13.652&0.025&13.734&0.050&-0.232&14.001&0.116&-0.512&---&---&-1.650&0.190 \\
184&13:27:28.50&-47:31:35.4&RRc&0.303&13.778&0.012&13.624&0.028&13.536&0.019&---&---&---&---&---&---&---&---&---&--- \\
185&13:26:04.13&-47:21:45.0&RRc&0.333&13.701&0.016&13.545&0.018&13.508&0.023&13.496&0.036&-0.043&13.479&0.033&-0.046&---&---&---&--- \\
261&13:27:15.41&-47:21:29.5&RRc&0.403&13.431&0.009&13.212&0.019&13.113&0.020&---&---&---&---&---&---&---&---&-1.500&0.350 \\
263&13:26:13.13&-47:26:09.7&RRab&1.012&13.155&0.017&12.888&0.017&12.746&0.016&---&---&---&12.660&0.034&---&---&---&-1.730&0.190 \\
274&13:26:43.73&-47:22:48.2&RRc&0.311&13.828&0.011&13.758&0.023&13.650&0.022&---&---&---&---&---&---&---&---&---&--- \\
276&13:27:16.51&-47:33:17.6&RRc&0.308&13.727&0.021&13.614&0.046&13.533&0.024&---&---&---&---&---&---&---&---&---&--- \\
280&13:27:09.33&-47:23:05.7&RRc&0.282&13.951&0.012&13.905&0.026&13.816&0.029&---&---&---&---&---&---&---&---&---&--- \\
285&13:25:40.20&-47:34:48.4&RRc&0.329&13.687&0.017&13.504&0.027&13.503&0.015&---&---&---&13.358&0.074&---&---&---&---&--- \\
288&13:28:10.32&-47:23:47.8&RRc&0.295&13.809&0.011&13.719&0.016&13.635&0.019&---&---&---&---&---&---&---&---&---&--- \\
289&13:28:03.68&-47:21:27.9&RRc&0.308&13.743&0.013&13.618&0.015&13.584&0.022&---&---&---&---&---&---&---&---&---&--- \\
291&13:26:38.52&-47:33:28.0&RRc&0.334&13.674&0.018&13.518&0.044&13.444&0.026&---&---&---&---&---&---&---&---&---&--- \\
357&13:26:17.77&-47:30:23.4&RRc&0.298&13.692&0.027&13.468&0.064&13.468&0.045&13.462&0.044&0.120&13.375&0.041&0.204&---&---&-1.640&0.990 \\
\end{longtable}
}
\end{center}
\end{landscape}
\clearpage


\begin{comment}
\section{Extra Figures}
\label{sec:extra_figures}
\begin{figure}
\begin{center}
\includegraphics[width=80mm]{reworked_fitting_code/final_plots/deltadelta_3p6_4p5_spect.pdf}
\caption{$\Delta$3.6~$\mu$m vs. $\Delta$4.5~$\mu$m using spectroscopic metallicities}
\label{fig:deltadelta_spect}
\end{center}
\end{figure}

\begin{figure}
\begin{center}
\includegraphics[width=80mm]{reworked_fitting_code/final_plots/deltadelta_3p6_4p5_phot.pdf}
\caption{$\Delta$3.6~$\mu$m vs. $\Delta$4.5~$\mu$m using photometric metallicities}
\label{fig:deltadelta_phot}
\end{center}
\end{figure}
\end{comment}


%%%%%%%%%%%%%%%%%%%%%%%%%%%%%%%%%%%%%%%%%%%%%%%%%%


% Don't change these lines
\bsp	% typesetting comment
\label{lastpage}
\end{document}

% End of mnras_template.tex